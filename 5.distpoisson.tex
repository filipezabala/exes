\chapter{Distribuição de Poisson}  

\begin{enumerate}[resume]
%\setcounter{enumi}{1}

\item Sabendo que certo processo industrial produz, em média, $4\%$ de unidades defeituosas e considerando uma amostra ocasional de cem unidades, determine a probabilidade de se obter:
	\begin{enumerate}
	\item nenhuma unidade defeituosa;
	\item uma unidade defeituosa;
	\item mais de uma unidade defeituosa;
	\item ao menos uma unidade defeituosa;
	\item duas unidades defeituosas;
	\item mais de duas unidades defeituosas;
	\item ao menos duas unidades defeituosas;
	\item um número de unidades defeituosas no intervalo $[1 ; 2]$.
	\end{enumerate}
	
\item Sabendo que em certo processo industrial, em média, duas máquinas necessitam de conserto, determine a probabilidade de que o numero de máquinas que necessitem de conserto, num dia qualquer, seja:
	\begin{enumerate}
	\item igual a zero;
	\item igual a um;
	\item maior ou igual a um;
	\item maior do que um;
	\item igual a dois;
	\item maior do que dois;
	\item menor do que dois.
	\end{enumerate}

\item Sabendo que a central telefônica de certa cidade pode fazer, no máximo, 10 conexões por minuto e que a média de chamadas é de 180 por hora, determine a probabilidade da central telefonica receber num determinado minuto:
	\begin{enumerate}
	\item nenhuma solicitação de conexão;
	\item uma solicitação de conexão;
	\item mais de uma solicitação de conexão;
	\item ao menos uma solicitação de conexão;
	\item mais solicitações de conexão do que pode suportar.
	\end{enumerate}

\item Sabendo que em certo processo industrial a média mensal de acidentes pessoais é igual a 0,5, determine a probabilidade de que, ao longo de quatro meses, verifique-se:
	\begin{enumerate}
	\item nenhum acidente;
	\item um acidente;
	\item ao menos um acidente;
	\item mais de um acidente;
	\item dois acidentes;
	\item no máximo dois acidentes.	
	\end{enumerate}	
	
\end{enumerate}	
