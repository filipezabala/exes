\chapter{Distribuição Normal}

\begin{enumerate}[resume]
%\setcounter{enumi}{1}

\item Sendo Z uma variável aleatória contínua com distribuição $N(0 ; 1)$, determine a: 
	\begin{enumerate}
	\item $P(0 < Z < 1)$;
	\item $P(0 < Z < 1,35)$;
	\item $ P(-2,33 < Z < 1,96)$;
	\item $P(0,65 < Z < 1,96) $;
	\item $P(-2,33 < Z < -0,65)$;
	\item $ P(-1,96 < Z < 1,96)$;
	\item $P(Z < -2,33 ou Z > 1,96)$;
	\item $P(Z < -1,96 ou Z > 2,33)$.
	\end{enumerate}

\item Sendo Z uma variável aleatória contínua com distribuição $N(0 ; 1)$, determine os valores de Z' que satisfazem as seguintes condições: 
	\begin{enumerate}
	\item $P(0 < Z < Z')=0,4772$;
	\item $P(-Z' < Z < 0) = 0,4452$;
	\item $ P(1,20 < Z < Z') = 0,1019$;
	\item $P(-1,20 < Z < Z') = 0,7061$;
	\item $P(-2,21 < Z < -Z') 0,3236$;
	\item $ P(Z < -1,96 ou Z > Z') = 0,0478$;
	\item $P(Z < -1,00 ou Z > Z') = 0,1837$.
	\end{enumerate}
	
\item Sendo X uma variável aleatória contínua com distribuição $N(10 ; 2)$, determine os valores de X' que satisfazem as seguintes condições: 
	\begin{enumerate}
	\item $P(10 < X < X')=0,4772$;
	\item $P(8 < X < X') = 0,3413$;
	\item $P(6 < X < X') = 0,1359$;
	\item $P(11 < X < X') = 0,2417$;
	\item $P(X > X') = 0,1251$;
	\item $P(X > X') = 0,8869$;
	\item $P(X > X') = 0,1020$;
	\item $P(X < X') = 0,1075$;
	\item $P(X < X') = 0,7996$;	
	\item $P(X < X') = 0,1112$.
	\end{enumerate}

\item Sabendo que o peso, dado em quilogramas, dos estudantes de certa escola está distribuido segundo uma $N(80 ; 5)$, determine o percentual de alunos que pesam:
	\begin{enumerate}
	\item entre 76 e 85 quilogramas;
	\item entre 82 e 86 quilogramas;
	\item entre 76 e 78 quilogramas;
	\item mais do que 82 quilogramas;
	\item mais do que 78 quilogramas;
	\item menos do que 82 quilogramas;
	\item menos do que 78 quilogramas;
	\item menos do que 75 ou mais do que 90 quilogramas.
	\end{enumerate}


\item Sabendo que a estatura, dada em centímetros, dos 2.000 estudantes de certa escola está distribuída segundo uma $N(170 ; 10)$, determine o número de alunos com estatura:
	\begin{enumerate}
	\item entre 162 e 174 centímetros;
	\item entre 174 e 180 centímetros;
	\item entre 150 e 168 centímetros;
	\item maior do que 174 centímetros;
	\item maior do que 165 centimetros;
	\item menor do que 160 centímetros;
	\item menor do que 180 centímetros;
	\item menor do que 160 ou maior do que 180 centímetros.
	\end{enumerate}

\item Considerando que o peso, dado em gramas, de determinado artigo produzido por uma fábrica seja $N(20 ; 4)$, determine:
	\begin{enumerate}
	\item a probabilidade de uma unidade, selecionada ao acaso, pesar entre 16 e 22 gramas;
	\item a probabilidade de uma unidade, selecionada ao acaso, pesar entre 22 e 25 gramas;
	\item a probabilidade de uma unidade, selecionada ao acaso, pesar mais de 23 gramas;
	\item a probabilidade de uma unidade, selecionada ao acaso, pesar menos de 16 gramas;
	\item o intervalo, centrado na média, dado em gramas, para o qual corresponda a probabilidade de 0,899 de que o peso de uma unidade selecionada ao acaso nele esteja contida;
	\item o peso, dado em gramas, abaixo do qual se espera encontrar $25,78\%$ das unidades produzidas;
	\item o peso, dado em gramas, acima do qual se espera encontrar $34,46\%$ das unidades produzidas;
	\item supondo a extração de uma amostra aleatória de 500 unidades, quantas se espera que pesem entre 15 e 21 gramas;
	\item supondo a extração de uma amostra aleatória de 1000 unidades, quantas se espera que pesem mais de 24 gramas;
	\item o peso médio das unidades produzidas pela máquina do processo industrial mencionado, que deve ser adotado, para que apenas $2,28\%$ das unidades pesem menos do que 12 gramas;
	\item a probabilidade de uma unidade, selecionada ao acaso, pesar menos do que 18 ou mais do que 24 gramas;
	\item a probabilidade de uma unidade, selecionada ao acaso, pesar menos do que 16 ou mais do que 24 gramas.
	\end{enumerate}

\item Sabendo que os quocientes de liquidez normal das empresas de uma determinada região estão distribuídos segundo uma $N( 1,10 ; 0,08)$, determine:
	\begin{enumerate}
	\item a probabilidade de que uma empresa, selecionada ao acaso, possua um quociente de liquidez entre 1,02 e 1,26;
	\item a probabilidade de que uma empresa, selecionada ao acaso, possua um quociente de liquidez entre 1,18 e 1,26;
	\item a probabilidade de que uma empresa, selecionada ao acaso, possua um quociente de liquidez maior do que 1,30;
	\item a probabilidade de que uma empresa, selecionada ao acaso, possua um quociente de liquidez menor do que 1,26;
	\item o intervalo, centrado na média, dado em termos de quociente de liquidez, para o qual corresponda a probabilidade de 0,8664 de que o quociente de liquidez de uma empresa selecionada ao acaso nele esteja contido;
	\item o quociente de liquidez normal abaixo do qual se espera encontrar $10,56\%$ das empresas;
	\item o quociente de liquidez normal acima do qual se espera encontrar $1,22\%$ das empresas;
	\item sabendo que na região existem 5.000 empresas e que a rede bancária aceita o desconto de títulos somente das que possuem um quociente de liquidez normal igual ou superior a 1,04, qual o numero provável das empresas que utilizam a mencionada operação;
	\item sabendo que a Secretaria da Receita Federal resolveu efetuar um exame detalhado das empresas cujos quocientes de liquidez normal se afastam do quociente médio duas e duas e meia unidades de desvio padrão, respectivamente, para menos e para mais, qual o número de inspeçoes a serem realizadas.
	\end{enumerate}
\end{enumerate}