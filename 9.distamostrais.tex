\chapter{Distribuições Amostrais}

\begin{enumerate}[resume]
%\setcounter{enumi}{1}

\item Sabendo que a característica estudada em certa população constituída de quatro unidades apresenta as determinações:
	\begin{table}[!htb]
	\centering
	\vspace{0.5cm}
	\begin{tabular}{ll}
	UNIDADE & $X_{i}$ \\
	\hline 
	$U_{1}$ & 24 \\
	$U_{2}$ & 30 \\
	$U_{3}$ & 36 \\
	$U_{4}$ & 42 \\
	\end{tabular}
	\end{table}
	\begin{enumerate}
	
	\item Determine as seguintes medidas na população investigada:
		\begin{enumerate}
		\item total;
		\item média aritmética; 
		\item variância absoluta;
		\item desvio padrão;
		\item variância relativa;
		\item coeficiente de variabilidade.
		\end{enumerate}
	
	\item Fixado o tamanho da amostra em duas unidades e admitindo o esquema de seleção sem reposição, determine:
		\begin{enumerate}
		\item as possíveis amostras;
		\item a distribuição amostral de média e calcule a sua expectancia, variância absoluta, desvio padrão, variância relativa e coeficiente de variabilidade;
		\item a distribuição amostral do total da amostra e calcule a sua expectância;
		\item a distribuição amostral do $\hat{T} = N\bar{X}$ e calcule a sua expectância, variância absoluta, desvio padrão, variância relativa e coeficiente de variabilidade;
		\item a distribuição amostral da variância absoluta e calcule a sua expectância;
		\item a distribuição amostral do $\hat{\sigma}^{2}$ e calcule a sua expectância.
		\end{enumerate}
	
	\item Fixado o tamanho da amostra em três unidades e admitindo o esquema de seleção sem reposição, determine:
		\begin{enumerate}
		\item as possíveis amostras;
		\item a distribuição amostral de média e calcule a sua expectancia, variância absoluta, desvio padrão, variância relativa e coeficiente de variabilidade;
		\item a distribuição amostral do total da amostra e calcule a sua expectância;
		\item a distribuição amostral do $\hat{T} = N\bar{X}$ e calcule a sua expectância, variância absoluta, desvio padrão, variância relativa e coeficiente de variabilidade;
		\item a distribuição amostral da variância absoluta e calcule a sua expectância;
		\item a distribuição amostral do $\hat{\sigma}^{2}$ e calcule a sua expectância.
		\end{enumerate}
		
			\item Considerando as distribuições de amostragem dos estimadores $\bar{X}$, $\hat{T}$ e $\hat{\sigma}^{2}$ para os tamanhos de amostra dois e três, determine:
		\begin{enumerate}
		\item a amplitude total de cada distribuição, comparando as que se referem ao mesmo estimador;
		\item o desvio padrão de cada distribuição, comparando os que se referem as distribuições do mesmo estimador.
		\end{enumerate}
	\end{enumerate}

\item Sabendo que a característica estudada em certa população constituída de quatro unidades apresenta as determinações:
	\begin{table}[!htb]
	\centering
	\vspace{0.5cm}
	\begin{tabular}{ll}
	UNIDADE & $X_{i}$ \\
	\hline 
	$U_{1}$ & 48 \\
	$U_{2}$ & 54 \\
	$U_{3}$ & 60 \\
	$U_{4}$ & 66 \\
	\end{tabular}
	\end{table}
	\begin{enumerate}

	\item Determine as seguintes medidas na população investigada:
		\begin{enumerate}
		\item total;
		\item média aritmética;
		\item variância absoluta;
		\item desvio padrão;
		\item variância relativa;
		\item coeficiente de variabilidade;
		\end{enumerate}
	
	\item Fixado o tamanho da amostra em duas unidades e admitido o esquema de seleção sem reposição, determine:
		\begin{enumerate}
		\item as possíveis amostras;
		\item a distribuição amostral da média e calcule a sua expectância, variância absoluta, desvio padrão, variância relativa e coeficiente de variabilidade;
		\item a distribuição amostral do total da amostra e calcule a sua expectância;
		\item a distribuição amostral do $\hat{T} = N\bar{X}$ e calcule a sua expectância, variância absoluta, desvio padrão, variância relativa e coeficiente de variabilidade;
		\item a distribuição amostral da variância absoluta e calcule a sua expectância;
		\item a distribuição amostral do $\hat{\sigma}^{2}$ e calcule a sua expectância.
		\end{enumerate}

	\item Fixado o tamanho da amostra em três unidades e admitido o esquema de seleção sem reposição, determine:
		\begin{enumerate}
		\item as possíveis amostras;
		\item a distribuição amostral da média e calcule a sua expectância, variância absoluta, desvio padrão, variância relativa e coeficiente de variabilidade;
		\item a distribuição amostral do total da amostra e calcule a sua expectância;
		\item a distribuição amostral do $\hat{T} = N\bar{X}$ e calcule a sua expectância, variância absoluta, desvio padrão, variância relativa e coeficiente de variabilidade;
		\item a distribuição amostral da variância absoluta e calcule a sua expectância;
		\item a distribuição amostral do $\hat{\sigma}^{2}$ e calcule a sua expectância.
		\end{enumerate}

	\item Considerando as distribuições de amostragem dos estimadores $\bar{X}$, $\hat{T}$ e $\hat{\sigma}^{2}$ para os tamanhos de amostra dois e três, determine:
		\begin{enumerate}
		\item a amplitude total de cada distribuição, comparando as que se referem ao mesmo estimador;
		\item o desvio padrão de cada distribuição, comparando os que se referem as distribuições do mesmo estimador;
		\item o coeficiente de variabilidade de cada distribuição, comparando os que se referem as distribuições do mesmo estimador.
		\end{enumerate}
	\end{enumerate}	

\item Sabendo que a característica estudada em certa população constituída de quatro unidades apresenta as determinações:
	\begin{table}[!htb]
	\centering
	\vspace{0.5cm}
	\begin{tabular}{ll}
	UNIDADE & $X_{i}$ \\
	\hline 
	$U_{1}$ & 12 \\
	$U_{2}$ & 18 \\
	$U_{3}$ & 24 \\
	$U_{4}$ & 30 \\
	\end{tabular}
	\end{table}
	
	\begin{enumerate}
	\item Determine as seguintes medidas na população investigada:
		\begin{enumerate}
		\item total;
		\item média aritmética;
		\item variância absoluta;
		\item desvio padrão;
		\item variância relativa;
		\item coeficiente de variabilidade;
		\end{enumerate}
	
	\item Fixado o tamanho da amostra em duas unidades e admitido o esquema de seleção sem reposição, determine:
		\begin{enumerate}
		\item as possíveis amostras;
		\item a distribuição amostral da média e calcule a sua expectância, variância absoluta, desvio padrão, variância relativa e coeficiente de variabilidade;
		\item a distribuição amostral do total da amostra e calcule a sua expectância;
		\item a distribuição amostral do $\hat{T} = N\bar{X}$ e calcule a sua expectância, variância absoluta, desvio padrão, variância relativa e coeficiente de variabilidade;
		\item a distribuição amostral da variância absoluta e calcule a sua expectância;
		\item a distribuição amostral do $\hat{\sigma}^{2}$ e calcule a sua expectância.
		\end{enumerate}

	\item Fixado o tamanho da amostra em três unidades e admitido o esquema de seleção sem reposição, determine:
		\begin{enumerate}
		\item as possíveis amostras;
		\item a distribuição amostral da média e calcule a sua expectância, variância absoluta, desvio padrão, variância relativa e coeficiente de variabilidade;
		\item a distribuição amostral do total da amostra e calcule a sua expectância;
		\item a distribuição amostral do $\hat{T} = N\bar{X}$ e calcule a sua expectância, variância absoluta, desvio padrão, variância relativa e coeficiente de variabilidade;
		\item a distribuição amostral da variância absoluta e calcule a sua expectância;
		\item a distribuição amostral do $\hat{\sigma}^{2}$ e calcule a sua expectância.
		\end{enumerate}
   	
		\item Considerando as distribuições de amostragem dos estimadores $\bar{X}$, $\hat{T}$ e $\hat{\sigma}^{2}$ para os tamanhos de amostra dois e três, determine:
		\begin{enumerate}   	
   		\item a amplitude total de cada distribuição, comparando as que se referem ao mesmo estimador;
		\item o desvio padrão de cada distribuição, comparando os que se referem as distribuições do mesmo estimador;
		\item o coeficiente de variabilidade de cada distribuição, comparando os que se referem as distribuições do mesmo estimador.
		\end{enumerate}
	\end{enumerate}
	
\item Supondo que tivesse sido proposto investigar a população do item 63 mediante um levantamento por amostragem, utilizando uma amostra de tamanho três, e que tenham sido sorteadas as unidades $U_{1}$, $U_{2}$ e $U_{4}$, determine:
	\begin{enumerate}
	\item a estimativa da média da população;
	\item a estimativa do total da população;
	\item a estimativa da variância absoluta da população;
	\item a estimativa da variância relativa da população;
	\item a estimativa do desvio padrão e do coeficiente de variabilidade da correspondente distribuição amostral da média;
	\item a estimativa do desvio padrão e do coeficiente de variabilidade da correspondente distribuição amostral de $\hat{T}$.
	\end{enumerate}
	
\item Supondo que tivesse sido proposto investigar a população do item 64 mediante um levantamento por amostragem, utilizando uma amostra de tamanho dois, e que tenham sido sorteadas as unidades $U_{2}$ e $U_{4}$, determine:
	\begin{enumerate}
	\item a estimativa da média da população;
	\item a estimativa do total da população;
	\item a estimativa da variância absoluta da população;
	\item a estimativa da variância relativa da população;
	\item a estimativa do desvio padrão e do coeficiente de variabilidade da correspondente distribuição amostral da média;
	\item a estimativa do desvio padrão e do coeficiente de variabilidade da correspondente distribuição amostral de $\hat{T}$.
	\end{enumerate}
	
\item Supondo que tivesse sido proposto investigar a população do item 65 mediante um levantamento por amostragem, utilizando uma amostra de tamanho três, e que tenham sido sorteadas as unidades $U_{1}$, $U_{3}$ e $U_{4}$, determine:
	\begin{enumerate}
	\item a estimativa da média da população;
	\item a estimativa do total da população;
	\item a estimativa da variância absoluta da população;
	\item a estimativa da variância relativa da população;
	\item a estimativa do desvio padrão e do coeficiente de variabilidade da correspondente distribuição amostral da média;
	\item a estimativa do desvio padrão e do coeficiente de variabilidade da correspondente distribuição amostral de $\hat{T}$.
	\end{enumerate}	
	
\item A população dos 2.000 trabalhadores da região ABC deve ser estudada em relação a sua renda anual, mediante um levantamento por amostragem, segundo o esquema de amostragem aleatória simples. Considerando que a investigação das unidades selecionadas na amostra tenha propiciado a elaboração da:
	\begin{table}[!htb]
	\centering
	\caption{Renda Anual de Uma Amostra de Trabalhadores da Região Sul, 1973}
	\vspace{0.5cm}
	\begin{tabular}{ll}
	Unidades Monetárias & N$^o$ de Trabalhadores \\
	\hline 
	25 $\vdash$ 35 & 20 \\
	35 $\vdash$ 45 & 40 \\
	45 $\vdash$ 55 & 25 \\
	55 $\vdash$ 65 & 10 \\
	65 $\vdash$ 75 & 5 \\
	\hline
	$\sum$ & 100
	\end{tabular}
	 \newline \newline Fonte: dados hipotéticos
	\end{table}
	\newline determine a estimativa:
	\begin{enumerate}
	\item da renda anual média da população;
	\item da renda anual total da população;
	\item da variância absoluta da renda anual da população;
	\item da variância relativa da renda anual da população;
	\item do desvio padrão e do coeficiente de variabilidade da correspondente distribuição amostral da média;
	\item do desvio padrão e do coeficiente de variabilidade da correspondente distribuição amostral de $\hat{T}$.
	\end{enumerate}	
	
\item A Industrial ABC S/A, fabricantes de pilhas elétricas, deseja avaliar a qualidade de seu produto quanto ao tempo de duração, dado em horas, mediante um levantamento por amostragem.Considerando que a investigação das unidades selecionadas na amostra tenha propiciado a elaboração da:
	\begin{table}[!htb]
	\centering
	\caption{Tempo de duração de 200 pilhas elétricas produzidas pela Industrial ABC S/A (h), 1973}
	\vspace{0.5cm}
	\begin{tabular}{ll}
	Horas & N$^o$ de Pilhas \\
	\hline 
	40 $\vdash$ 44 & 20 \\
	44 $\vdash$ 48 & 40 \\
	48 $\vdash$ 52 & 80 \\
	52 $\vdash$ 56 & 40 \\
	56 $\vdash$ 60 & 20 \\
	\hline
	$\sum$ & 200
	\end{tabular}
	 \newline \newline Fonte: dados hipotéticos
	\end{table}
	\newline determine a estimativa:
	\begin{enumerate}
	\item da média do tempo de duração da população;;
	\item da variância absoluta do tempo de duração da população;
	\item da variância relativa do tempo de duração da população;
	\item do desvio padrão e do coeficiente de variabilidade da correspondente distribuição amostral da média.
	\end{enumerate}		

\item Sabendo que o peso, dado em gramas, de determinado artigo produzido por uma fábrica esteja distribuído segundo uma $N(400 ; 40)$, determine
	\begin{enumerate}
	\item a probabilidade de uma unidade, selecionada ao acaso, estar contida no intervalo $[390 ; 410]$;
	\item a probabilidade de uma unidade, selecionada ao acaso, pesar menos do que 392 ou mais do que 408 gramas;
	\item considerando a extração de uma amostra aleatória de 25 unidades, a probabilidade de sua média estar contida no intervalo $[390 ;410]$;
	\item considerando a extração de uma amostra aleatória de 25 unidades, a probabilidade de sua média pesar menos do que 392 ou mais do que 408 gramas;
	\item considerando a extração de uma amostra aleatória de 100 unidades, a probabilidade de sua média estar contida no intervalo $[390 ;410]$;
	\item considerando a extração de uma amostra aleatória de 100 unidades, a probabilidade de sua média pesar menos do que 392 ou mais do que 408 gramas;
	\item considerando a extração de uma amostra aleatória de 100 unidades, a probabilidade de que o peso total das unidades nela selecionadas seja menor do que 40.400 gramas.
	\end{enumerate}

\item Supondo que o tempo de duração, dado em horas ,de certo dispositivo eletrônico esteja distribuído segundo uma $N(500 ; 20)$, determine
	\begin{enumerate}
	\item a probabilidade do tempo de duração de uma unidade, selecionada ao acaso, estar contido no intervalo $[496 ; 504]$;
	\item a probabilidade do tempo de duração de uma unidade, selecionada ao acaso, ser menor do que 498 ou maior do que 503 horas;
	\item considerando a extração de uma amostra aleatória de 25 unidades, a probabilidade de seu tempo médio de duração estar contido no intervalo $[496 ; 504]$;
	\item considerando a extração de uma amostra aleatória de 25 unidades, a probabilidade de seu tempo médio de duração ser menor do que 498 ou maior do que 503 horas;
	\item considerando a extração de uma amostra aleatória de 64 unidades, a probabilidade de seu tempo médio de duração estar contido no intervalo $[496 ; 504]$;
	\item considerando a extração de uma amostra aleatória de 64 unidades, a probabilidade de seu tempo médio de duração ser menor do que 498 ou maior do que 503 horas;
	\item considerando a extração de uma amostra aleatória de 64 unidades, a probabilidade de que o total do tempo de duração das unidades nela selecionadas seja maior do que 32.192 horas.
	\end{enumerate}														

\item Sabendo que o peso, dado em gramas, de certo artigo produzido por uma fábrica possui média 800 e desvio padrão 12, determine:
	\begin{enumerate}
	\item considerando a extração de uma amostra ocasional de 36 unidades, a probabilidade de sua média assumir um valor no intervalo $[796 ; 804]$;
	\item considerando a extração de uma amostra ocasional de 64 unidades, a probabilidade de sua média assumir um valor no intervalo $[796 ; 804]$;
	\item considerando a extração de uma amostra ocasional de 100 unidades, a probabilidade de sua média assumir um valor no intervalo $[796 ; 804]$.
	\end{enumerate}

\item Supondo que a média aritmética e o desvio padrão dos salários dos 2.000 empregados de certa empresa correspondam a 400 e 64 unidades monetária e considerando uma amostra ocasional de 256 operários, determine:
	\begin{enumerate}
	\item a probabilidade de sua média estar contida no intervalo $[396 ; 404]$;
	\item a probabilidade de sua média assumir um valor menor do que 398 ou maior do que 402 unidades monetárias;
	\item a probabilidade do total dos salários dos operários dela integrantes ser menor do que 101.888 unidades monetárias;
	\item a probabilidade do $\hat{T} = N\bar{X}$ estar contido no intervalo $[792.000 ; 806.000]$;
	\item a probabilidade do $\hat{T} = N\bar{X}$ assumir um valor menor do que 796.000 ou maior do que 804.000 unidades monetárias.
	\end{enumerate}
	
\item Supondo que a média aritmética e o desvio padrão dos salários dos 2.000 empregados de certa empresa correspondam a 400 e 64 unidades monetárias e considerando uma amostra ocasional de 400, determine:
	\begin{enumerate}
	\item a probabilidade de sua média estar contida no intervalo $[396 ; 404]$;
	\item a probabilidade de sua média assumir um valor menor do que 398 ou maior do que 402 unidades monetárias;
	\item a probabilidade do total dos salários dos operários dela integrantes ser menor do que 160.800 unidades monetárias;
	\item a probabilidade do $\hat{T} = N\bar{X}$ estar contido no intervalo $[792.000 ; 806.000]$;
	\item a probabilidade do $\hat{T} = N\bar{X}$ assumir um valor menor do que 796.000 ou maior do que 804.000 unidades monetárias.
	\end{enumerate}

\item Supondo que a média aritmética e o desvio padrão dos 2.000 empregados de certa empresa correspondam a 400 e 64 unidades monetárias e considerando uma amostra ocasional de 100 operários, determine:
	\begin{enumerate}
	\item a probabilidade de sua média estar contida no intervalo $[396 ; 404]$;
	\item a probabilidade de sua média assumir um valor menor do que 398 ou maior do que 402 unidades monetárias;
	\item a probabilidade do total dos salários dos operários dela integrantes ser menor do que 39.600 unidades monetárias;
	\item a probabilidade do $\hat{T} = N\bar{X}$ estar contido no intervalo $[792.000 ; 806.000]$;
	\item a probabilidade do $\hat{T} = N\bar{X}$ assumir um valor menor do que 796.000 ou maior do que 804.000 unidades monetárias.
	\end{enumerate}

\item Sabendo que as unidades de certa população investigadas em relação a determinado atributo permitiram verificar que:
	\begin{table}[!htb]
	\centering
	\vspace{0.5cm}
	\begin{tabular}{ll}
	UNIDADE & ATRIBUTO \\
	\hline 
	$U_{1}$ & $\bar{A}$ \\
	$U_{2}$ & A \\
	$U_{3}$ & $\bar{A}$ \\
	$U_{4}$ & A \\
	\end{tabular}
	\end{table} \\	
	\begin{enumerate}
	\item determine a proporção das unidades da população que apresenta o atributo A;
	\item considerando as possíveis amostras de tamanho 2 e 3, admitido o esquema de seleção sem reposição, determine a respectiva distribuição amostral das proporções e calcule a expectância, a variância absoluta e relativa correspondentes.
	\end{enumerate}
	
\item Sabendo que as unidades de certa população investigadas em relação a determinado atributo permitiram verificar que:
	\begin{table}[!htb]
	\centering
	\vspace{0.5cm}
	\begin{tabular}{ll}
	UNIDADE & ATRIBUTO \\
	\hline 
	$U_{1}$ & A \\
	$U_{2}$ & $\bar{A}$ \\
	$U_{3}$ & A  \\
	$U_{4}$ & $\bar{A}$ \\
	$U_{5}$ & A \\
	\end{tabular}
	\end{table} \\	
	\begin{enumerate}
	\item determine a proporção das unidades da população que apresenta o atributo A;
	\item considerando as possíveis amostras de tamanho 2, 3 e 4, admitido o esquema de seleção sem reposição, determine a respectiva distribuição amostral das proporções e calcule a expectância, a variância absoluta e relativa correspondentes.
	\end{enumerate}	
	
\item Sabendo que $80\%$ das unidades produzidas por determinada fábrica são, face às caracteristicas apresentadas, classificadas como artigo de exportação, determine:
	\begin{enumerate}
	\item considerando a extração de uma amostra aleatória de 64 unidades, a probabilidade de que a proporção de unidades classificadas como artigo de exportação assuma um valor no intervalo $[0,76 ; 0,86]$;
	\item considerando a extração de uma amostra aleatória de 100 unidades, a probabilidade de que a proporção de unidades classificadas como artigo de exportação assuma um valor no intervalo $[0,76 ; 0,86]$;
	\item considerando a extração de uma amostra aleatória de 400 unidades, a probabilidade de que a proporção de unidades classificadas como artigo de exportação assuma um valor no intervalo $[0,76 ; 0,86]$;
	\item considerando a extração de uma amostra aleatória de 100 unidades, a probabilidade de que a proporção de unidades classificadas como artigo de exportação assuma um valor maior do que 0,90;
	\item considerando a extração de uma amostra aleatória de 100 unidades, a probabilidade de que a proporção de unidades classificadas como artigo de exportação assuma um valor menor do que 0,74;
	\end{enumerate}

\end{enumerate}