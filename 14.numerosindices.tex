\chapter{Números Índices}

\begin{enumerate}[resume]
%\setcounter{enumi}{1}

\item Considerando os dados constantes da tabela abaixo, fixado o período base em 1974, determine, interpretando os resultados obtidos, os seguinte índices de preços:
	%\begin{enumerate}
	%\item agregativo simples;
	%\item média aritmética dos preços relativos;
	%\item agregativo ponderado de Laspeyres;
	%\item	 agregativo ponderado de Paasche;
	%\item agregativo ponderado de Fisher.
	%\end{enumerate}
	%\begin{table}[!htb]
	%\begin{tabular}{|l|l|l|l|l|l|l|l|l|}
%	\hline
%	\multicolumn{1}{|c|}}{\multirow{2}{}{utilidades}} & \multicolumn{2}{c|}{1974} & \multicolumn{2}{c|}{1975} & \multicolumn{2}{c|}{1976} & \multicolumn{2}{c|}{1977} \\ \cline{2-9} 
%	\multicolumn{1}{|c|}{} & preços & quant. & preços & quant. & preços & quant. & preços & quant. \\ \hline
%	A & 4 & 10 & 6 & 9 & 11 & 8 & 12 & 6 \\ \hline
%	B & 10 & 6 & 9 & 8 & 7 & 9 & 5 & 12 \\ \hline
%	C & 5 & 8 & 7 & 9 & 8 & 10 & 15 & 12 \\ \hline
%	D & 12 & 10 & 11 & 9 & 10 & 8 & 7 & 7 \\ \hline
%	E & 5 & 12 & 10 & 12 & 12 & 12 & 15 & 12 \\ \hline
%	F & 12 & 5 & 11 & 5 & 9 & 5 & 7 & 5 \\ \hline
%	G & 8 & 5 & 8 & 6 & 8 & 7 & 8 & 8 \\ \hline
%	H & 8 & 10 & 8 & 8 & 8 & 7 & 8 & 5 \\ \hline
%	\end{tabular}
%	\\ Fonte: dados hipotéticos
%	\end{table}
	
\item Considerando os dados constantes da tabela abaixo, fixado o período base em 1972, determine, interpretando os resultados obtidos, os seguinte índices de preços:
	\begin{enumerate}
	\item agregativo simples;
	\item média aritmética dos preços relativos;
	\item agregativo ponderado de Laspeyres;
	\item	 agregativo ponderado de Paasche;
	\item agregativo ponderado de Fisher.
	\end{enumerate}
%	\begin{table}[!htb]
%	\begin{tabular}{|l|l|l|l|l|l|l|}
%	\hline
%	\multicolumn{1}{|c|}}{\multirow{2}{}{utilidades}} & \multicolumn{2}{c|}{1971} & \multicolumn{2}{c|}{1972} & \multicolumn{2}{c|}{1973} \\ \cline{2-9} 
%	\multicolumn{1}{|c|}{} & preços & quant. & preços & quant. & preços & quant. \\ \hline
%	A & 30 & 8 & 20 & 10 & 25 & 12 \\ \hline
%	B & 20 & 10 & 24 & 10 & 24 & 8 \\ \hline
%	C & 10 & 10 & 10 & 8 & 10 & 10  \\ \hline
%	D & 6 & 5 & 5 & 6 & 6 & 6  \\ \hline
%	E & 5 & 4 & 6 & 5 & 8 & 6 \\ \hline
%	\end{tabular}
%	\\ Fonte: dados hipotéticos
%	\end{table}

\item Considerando os dados da tabela abaixo, determine os índices de preços de Laspeyres, Paasche e Fisher, fixada a base em 1972, e interprete os resultados obtidos.
%	\begin{table}[!htb]

%	\end{table}

\item Considerando os dados da tabela abaixo, determine os índices de preços de Laspeyres, Paasche e Fisher, fixada a base em 1972, e interprete os resultados obtidos.
%	\begin{table}[!htb]

%	\end{table}

\end{enumerate}