\chapter{Estimação}

\begin{enumerate}[resume]
%\setcounter{enumi}{1}

\item De uma população composta de 2.000 pessoas, cuja distribuição de salários apresenta $\sigma^{2}$  $= 4.497,75$, foi extraída uma amostra aleatória de tamanho n = 400. Sabendo que a média da amostra é de 900 unidades monetárias e considerando um nível de significância de 0,05, estabeleça o intervalo de estimação da média e do total dos salários da população.

\item De uma população composta de 1.000 pessoas, cuja distribuição de salários apresenta $\gamma^{2}$ $= 0,25$ extraída uma amostra aleatória de tamanho n = 200. Sabendo que a média da amostra e de 500 unidades monetárias e considerando-se um nível de confiança de 0,95, estabeleça o intervalo de estimação da média e do total da população.

\item Da populaçao dos 400 alunos matriculados na cadeira de estatística em determinado curso universitário, cuja distribuição de notas apresenta o $\sigma^{2}$  $=225$, foi extraída uma amostra aleatória de tamanho n = 36. Sabendo que a média da amostra e de 70 pontos e considerando um nível de significância de 0,05, estabeleça o intervalo de estimação da média da população.

\item  De uma população composta de 4.000 famílias, cuja distribuição de salários apresenta $\gamma^{2}$ $ =0, 36$ foi extraída uma amostra aleatória de tamanho n =100. Sabendo que a média da amostra é de 800 unidades monetárias e considerando um nível de confiança de 0,95, estabeleça o intervalo de estimação da média e do total da população.

\item A Industrial ABC S/A, desejando conhecer o comprimento médio, dado em metros, das barras de ferro que produz, selecionou uma amostra aleatória de 40 unidades. Sabendo, face a experiência anterior, que a variância relativa da variável estudada é igual a 0,016 e que a média da referida amostra é de 4 metros, estabeleça, considerando um nível de confiança de 0,9544, o correspondente intervalo de estimação.

\item A Industrial XYZ S/A, fabricante de lâmpadas, desejando conhecer o tempo médio de duração de seu produto, selecionou uma amostra aleatória de 36 unidades. Sabendo, face a experiência anterior, que a variância absoluta da variável estudada é de 576 e que o tempo médio de duração apurado na referida amostra é de 600 horas, estabeleça, considerando um nível de confiança de 0,9544, o correspondente intervalo de estimação.

\item Sabendo que de uma populaçao infinita, na qual a variância relativa é igual a 0,25, foi extraída uma amostra ocasional de tamanho 100, tendo sido apurado que $\sum X_{i}= 2.500$, determine, ao nível de confiança de 0,95, os limites de confiança para a média populacional.

\item Sabendo que de uma população infinita, na qual a variância relativa é igual a 0,16, foi extraída uma amostra ocasional de tamanho 100, determine, ao nível de significância de 0,05, o erro relativo máximo de estimação associado a estimativa da média populacional da variável estudada.

\item Sabendo que de uma população de tamanho 401, na qual a variância absoluta é igual a 405, foi extraída uma amostra aleatória de tamanho 81, determine, ao nível de significância de 0,05, o erro absoluto máximo de estimação associado à estimativa da média populacional da variável estudada.

\item Sabendo que de uma população de tamanho 501, na qual a variancia relativa é igual a 0,051, foi extraída uma amostra aleatória de tamanho 51, determine, ao nível de significância de 0,05, o erro relativo máximo de estimação associado à estimativa da média populacional da variável.

\item Sabendo que o intervalo de estimaçao da média da variável X, característica de uma população infinita, foi estabelecido em $[180 ; 220]$, determine:
	\begin{enumerate}
	\item o valor da média da amostra investigada;
	\item o erro absoluto máximo de estimação associado à estimativa obtida;
	\item considerando que o esquema de amostragem foi estruturado ao nível de significância de 0,05, o valor do erro de amostragem na estimativa efetuada.
	\end{enumerate}

\item Sabendo que o intervalo de estimativa do total das determinações de uma variável X foi estabelecido em $[4.750 ; 5.250]$, determine:
	\begin{enumerate}
	\item o valor numérico do estimador não tendencioso do total da população investigada;
	\item o erro absoluto máximo de estimação associado à estimativa obtida;
	\item considerando que o esquema de amostragem foi estruturado ao nível de significância de 0,05, o valor do erro de amostragem na estimativa efetuada.
	\end{enumerate}
	
\item Sabendo que das 1.925 famílias de determinada região foi extraída uma amostra ocasional de 400 unidades, na qual foi apurado que 120 famílias possuíam veículo automotor, determine:
	\begin{enumerate}
	\item considerando o erro absoluto de estimação e um nível de confiança de 0,9544, o intervalo de estimativa da proporção de famílias que possue veículo automotor na referida região;
	\item considerando o erro relativo de estimação e um nível de confiança de 0,9544, o intervalo de estimativa da proporção de famílias que possue veículo automotor na referida região.
	\end{enumerate}		

\item Sabendo que das 2.500 famílias de determinada cidade foi extraída uma amostra aleatória de 500 unidades, na qual foi apurado que 160 famílias possuíam aparelho de televisão a cores, determine:
	\begin{enumerate}
	\item considerando o erro absoluto de estimação e um nível de confiança de 0,9544, o intervalo de estimativa da proporção de famílias que possue aparelho de televisão a cores na referida cidade;
	\item considerando o erro relativo de estimação e um nível de confiança de 0,9544, o intervalo de estimativa da proporção de famílias que possue aparelho de televisão a cores na referida cidade;
	\end{enumerate}

\item Sabendo que das 400 unidades investigadas de certo artigo industrial, 80 apresentaram defeito de fabricação, estabeleça, considerando o erro absoluto de estimação e um nível de confiança de 0,9544, o intervalo de estimativa da proporção de
unidades produzidas que apresentam defeito de fabricação.

\item Sabendo que das 100 unidades investigadas de certo artigo industrial, 10 apresentaram defeito de fabricação, estabeleça, considerando o erro absoluto de estimação e um nível de confiança de 0,9544, o intervalo de estimativa da proporção de
unidades produzidas que apresentam defeito de fabricação.

\item Sabendo que em uma população de tamanho 1.000 a variância absoluta da variável X é igual a 900 e considerando um nível de confiança de 0,95 e um erro absoluto de estimação de 10, determine:
	\begin{enumerate}
	\item o tamanho da amostra necessário à estimação de sua média populacional;
	\item supondo que na amostra selecionada verifique-se $\sum X_{i} = 17.000$, o correspondente intervalo de estimação.
	\end{enumerate}
	
\item Sabendo que em uma população de tamanho 1.000 a variância absoluta da variável X é igual a 900 e considerando um nível de confiança de 0,95 e um erro absoluto de estimação de 2.500, determine:
	\begin{enumerate}
	\item o tamanho da amostra necessário à estimação do seu total populacional;
	\item supondo que na amostra selecionada verifique-se $\sum X_{i} =  180.285$, o correspondente intervalo de estimação.
	\end{enumerate}

\item Sabendo que em uma população de tamanho 1.000 a variância relativa da variável X é igual a 0,16 e considerando um nível de confiança de 0,95 e um erro relativo de 0,10, determine:
	\begin{enumerate}
	\item o tamanho da amostra necessário à estimação do seu total e de sua média populacional;
	\item supondo que na amostra selecionada verifique-se $\sum X_{i} =  11.600$, os correspondentes intervalos de estimação.
	\end{enumerate}

\item Sabendo que em uma população infinita a variância absoluta da variável X é igual a 2.500 e considerando um nível de confiança de 0,9544 e um erro absoluto de estimação de 10, determine:
	\begin{enumerate}
	\item o tamanho da amostra necessário à estimação de sua média populacional;
	\item supondo que na amostra selecionada verifique-se $\sum X_{i} =  45.000$, o correspondente intervalo de estimação.	
	\end{enumerate}

\item Sabendo que em uma população infinita a variância relativa da variável X é igual a 0,25, e considerando um nível de confiança de 0,9544 e um erro relativo de 0,10, determine:
	\begin{enumerate}
	\item o tamanho da amostra necessário à estimação de sua média populacional;
	\item supondo que na amostra selecionada verifique-se $\sum X_{i} =  60.000$, o correspondente intervalo de estimação.	
	\end{enumerate}

\item Uma população infinita deve ser amostrada para se estabelecer uma estimativa da média populacional da variável X que possue, de acordo com estudos anteriores, variância relativa igual a 0,09. Considerando que o total dos recursos disponíveis para custear tal pesquisa é de 2.000 unidades monetárias, o seu custo fixo é de 200 unidades monetárias e o custo médio por unidade investigada é de 50 unidades monetárias, determine:
	\begin{enumerate}
	\item adotado o critério do custo fixo e erro variável, o tamanho da amostra;
	\item fixado o nível de confiança de 0,9544, o erro relativo de estimação decorrente do tamanho da amostra utilizado;	 
	\item supondo que na amostra selecionada verifique-se $\sum X_{i} =  7.200$, o correspondente intervalo de estimação.	
	\end{enumerate}

\item A Industrial ZHX S/A deseja conhecer o comprimento médio, dado em metros, das barras de ferro que produz. Sabendo, face a experiência anterior, que a variância relativa da variável estudada e, aproximadamente, 0,30 e considerando um nível de confiança de 0,9544 e um erro relativo de estimação de $10\%$, determine:
	\begin{enumerate}
	\item o tamanho mínimo da amostra necessário à estimação do comprimento médio das unidades produzidas;
	\item supondo que na referida amostra verifique-se que $\sum X_{i} = 720$, o correspondente intervalo de estimativa;
	\item considerando as condições anteriores, o erro relativo de estimação na hipótese de se utilizar uma amostra constituída de 80 unidades.
	\end{enumerate}
 
\item Um fabricante de tijolos deseja conhecer a qualidade de seu produto quanto a resistência ao rompimento. Sabendo, face a experiência anterior, que o desvio padrão da resistência ao rompimento é de 20 quilos e considerando um nível de confiança de 0,9544 e um erro absoluto de estimação de 5 quilos, determine:
	\begin{enumerate}
	\item o tamanho mínimo da amostra necessário à estimação desejada;
	\item supondo que na referida amostra verifique-se que $\sum X_{i} = 12.800$, o correspondente intervalo de estimativa;
	\item considerando as condições anteriores, o erro absoluto de estimação na hipótese de se utilizar uma amostra constituída de 100 unidades.
	\end{enumerate}

\item Com a finalidade de estimar a produção média e o total da produção de arroz de determinada área geográfica, onde existem 2.000 estabelecimentos produtores, deve ser efetuado um levantamento por amostragem. Sabendo que a variância absoluta da variaável estudada é de 2.025 e considerando um nível de confiança de 0,95 e um erro absoluto de 5 toneladas para a estimação da média, determine:
	\begin{enumerate}
	\item o tamanho mínimo da amostra necessário e suficiente à estimação desejada;
	\item supondo que que o custo fixo do planejamento e execução de tal pesquisa é de 1.600 unidades monetárias e que o custo médio por unidade investigada é de 25 unidades monetárias, o custo total da pesquisa;
	\item considerando que a produção total na amostra seja de 27.000 toneladas, os correspondentes intervalos de estimativa.
	\end{enumerate}

\item Com a finalidade de estimar a produção média e o total da produção de arroz de determinada área geográfica, onde existem 2.000 estabelecimentos produtores, deve ser efetuado um levantamento por amostragem. Sabendo que a variância relativa da produção dos estabelecimentos agrícolas e de 0,36 e considerando um nível de confiança de 0,9544 e um erro relativo máximo de $10\%$, determine:
	\begin{enumerate}
	\item o tamanho mínimo da amostra necessário e suficiente à estimação desejada;
	\item supondo que que o custo fixo do planejamento e execução de tal pesquisa é de 5.000 unidades monetárias e que o custo médio por estabelecimento investigado é de 40 unidades monetárias, o custo total da pesquisa;
	\item considerando que a produção total na amostra seja de 6.750 toneladas, os correspondentes intervalos de estimativa.
	\end{enumerate}

\item A Industrial ZBD S/A, fabricante de lâmpadas elétricas, deseja avaliar o tempo médio, dado em horas, de duração de seu produto. Considerando a inexistência de informações sobre a variável estudada, foi efetuada uma amostra piloto verificando-se que \\
$$n' = 20 \sum X_{i} = 1.960  \sum X_{i}^{2} = 208.502,8.$$\\
Sabendo que foi fixado, para a estimação desejada, um nível de confiança de 0,9544 e um erro relativo de $5\%$, determine o tamanho da amostra final.\\
Supondo, finalmente, que na amostra final apura-se:\\
$$\sum X_{i} = 9.360  \sum X_{i}^{2} = 656.733,6,$$\\
discuta a aceitabilidade do estimador utilizado no dimensionamento de n e determine o correspondente intervalo de estimação. \\
Na resolução do presente exercício, desenvolva os cálculos, procedendo o arredondamento devido, até a segunda decimal.

\item Com a finalidade de estimar a produção média e o total da produção de trigo, dada em toneladas, de determinada área geográfica, onde existem 3.000 estabelecimentos produtores, deve ser efetuado um levantamento por amostragem para o qual foi estabelecido um nível de confiança de 0,9544 e um erro mãximo de $10\%$. Considerando a inexistência de informações sobre a referida população, foi precedida uma amostra preliminar na qual \\
$$n' = 20 \sum X_{i} = 1.000  \sum X_{i}^{2} = 60.000.$$\\
Supondo, igualmente, que na amostra final verifique-se que:\\
$$\sum X_{i} = 4.428  \sum X_{i}^{2} = 286.426,$$\\
discuta a aceitabilidade do estimador utilizado no dimensionamento de n. Sabendo, finalmente, que o custo fixo do planejamento e execução de tal pesquisa e de 4.000 unidades monetárias e que o custo médio por unidade investigada é de 30 unidades monetárias, determine o custo total do levantamento.

\item Com a finalidade de estimar a produção média e o total da produção de trigo, dada em toneladas,de certa área geográfica, onde existem 4.000 estabelecimentos produtores, deve ser efetuado um levantamento por amostragem para o qual foi estabelecido um nível de confiança de 0,9544 e um erro máximo de $8\%$. Considerando a inexistência de informações sobre a referida população, foi precedida uma pesquisa preliminar na qual\\
$$n' = 30 \sum X_{i} = 2.400  \sum X_{i}^{2} = 219.853,8,$$\\
Supondo, igualmente, que na amostra final apure-se:\\
$$\sum X_{i} = 7.360  \sum X_{i}^{2} = 688.983,3,$$\\
e sabendo que o custo fixo do planejamento e execução da pesquisa é de 2.500 unidades monetárias e que o custo médio por unidade investigada é de 20 unidades monetárias, determine o custo total da pesquisa.\\
Na resolução do presente exercício, desenvolva os cálculos, procedendo o arredondamento devido, até a segunda decimal.

\item Com a finalidade de estimar o tempo médio de duração, dado em horas,das lâmpadas elétricas produzidas pela Industrial ABC S/A, deve ser efetuado um levantamento por amostragem para o qual foi fixado um nível de confiança de 0,9544 e um erro relativo máximo de $10\%$.\\
Considerando a inexistência de informações sobre a variável estudada, deve ser, inicialmente, realizada uma amostra piloto para a qual foi fixado um orçamento de 6.300 unidades monetárias.\\
Sabe-se, por outro lado, que o custo fixo de tal pesquisa preliminar é de 1.500 unidades monetárias e que o custo médio por unidade investigada é de 120 unidades monetárias.\\
Supondo que a investigação das unidades selecionadas na amostra piloto tenha proporcionado\\
$$\sum X_{i} = 2.000  \sum X_{i}^{2} = 115.600,$$\\
determine o correspondente tamanho da amostra final.\\
Considerando, finalmente, que na amostra final apure-se:\\
$$\sum X_{i} = 3.328  \sum X_{i}^{2} = 198.608,64,$$\\
discuta a aceitabilidade do estimador utilizado no dimensionamento de n e determine o correspondente intervalo de estimação.\\
Na resolução do presente exercício, desenvolva os cálculos, procedendo o arredondamento devido, até a segunda decimal.

\item Com a finalidade de estimar a produção média e o total da produção de soja de determinada região, onde existem 5.175 produtores, deve ser efetuado um levantamento por amostragem para o qual foi fixado um nível de confiança de 0,9544 e um erro
relativo máximo de $5\%$.\\
Considerando a inexistência de informações sobre a variável estudada, deve ser procedida uma pesquisa preliminar para a qual foi fixado um orçamento de 3.520 unidades monetárias.\\
Sabe-se, por outro lado, que o custo fixo de tal amostra piloto é de 2.360 unidades monetárias e que o custo médio por unidade investigada é de 58 unidades monetárias.\\
Supondo que a investigação das unidades selecionadas na pesquisa piloto tenha proporcionado \\
$$\sum X_{i} = 160  \sum X_{i}^{2} = 1.438$$\\
determine o correspondente tamanho da amostra final.\\
Considerando, finalmente, que a amostra final permita a elaboração da seguinte\\
	\begin{table}[!htb]
	\centering
	\caption{Produção de soja dos estabelecimentos selecionados na amostra (t)}
	\vspace{0.5cm}
	\begin{tabular}{ll}
	Toneladas & Frequência Relativa \\
	\hline 
	3 $\vdash$ 5 & 0,16  \\
	5 $\vdash$ 7 & 0,22  \\
	7 $\vdash$ 9 & 0,31  \\
	9 $\vdash$ 11 & 0,18  \\
	11 $\vdash$ 13 & 0,13  \\
	\end{tabular}
	 \newline \newline Fonte: dados hipotéticos
	\end{table}\\
discuta a aceitabilidade do estimador utilizado no dimensionamento da amostra e determine os correspondentes intervalos de estimação. \\
Na resolução do presente exercício, desenvolva os cálculos, procedendo o arredondamento devido, até a segunda decimal.

\item Com a finalidade de estimar a média aritmética da variável X de uma população infinita, deve ser efetuado um levantamento por amostragem para o qual foi fixado um nível de confiança de 0,9544 e um erro relativo de $10\%$. Considerando a inexistência de informações sobre a variavel estudada, deve ser, preliminarmente, realizada uma amostra piloto para a qual foi fixado um orçamento de 3.000 unidades monetárias.\\
Sabe-se, por outro lado, que o custo fixo de tal pesquisa preliminar é de 1.375 unidades monetárias e que o custo médio por unidade investigada é de 65 unidades monetárias.\\
Supondo que a investigação das unidades selecionadas na amostra piloto tenha proporcionado\\
$$\sum X_{i} = 75  \sum X_{i}^{2} = 268,25$$\\
determine o correspondente tamanho da amostra final.\\
Considerando, finalmente, que a amostra final permita a construção da seguinte\\
	\begin{table}[!htb]
	\centering
	\caption{Valores da variável X na amostra}
	\vspace{0.5cm}
	\begin{tabular}{ll}
	Classes & Frequência Relativa \\
	\hline 
	0,5 $\vdash$ 1,5 & 0,15  \\
	1,5 $\vdash$ 2,5 & 0,25  \\
	2,5 $\vdash$ 3,5 & 0,35  \\
	3,5 $\vdash$ 4,5 & 0,15 \\
	4,5 $\vdash$ 5,5 & 0,10  \\
	\end{tabular}
	\end{table}\\
discuta a aceitabilidade do estimador utilizado no dimensionamento da amostra e determine o correspondente intervalo de estimação. Na resolução do presente exercício, desenvolva desenvolva os cálculos, procedendo o arredondamento devido, até a segunda decimal.

\item Com a finalidade de estimar o salário médio e o total dos salários pagos em determinada região, onde existem 2.385 pessoas que percebem salário, deve ser efetuado um levantamento por amostragem para o qual foi fixado um nível de confiança de 0,9544 e um erro relativo máximo de 52.\\
Considerando a inexistência de informações sobre a variável estudada, deve ser procedida uma pesquisa preliminar para a qual foi fixado um orçamento de 2.850 unidades monetárias.\\
Sabe-se, por outro lado, que o custo fixo de tal amostra piloto é de 1.500 unidades monetárias e que o custo medio por unidade investigada é de 75 unidades monetárias.\\
Supondo que a investigação das unidades selecionadas na pesquisa piloto tenha proporcionado\\
$$\sum X_{i} = 180  \sum X_{i}^{2} = 1970,1$$\\
determine o correspondente tamanho da amostra final.\\
Considerando, finalmente, que a amostra final permita a elaboração da seguinte\\
	\begin{table}[!htb]
	\centering
	\caption{Salários pagos segundo os dados da amostra (U.M) na Região ABC}
	\vspace{0.5cm}
	\begin{tabular}{ll}
	Toneladas & Frequência Relativa \\
	\hline 
	5 $\vdash$ 7 & 0,18  \\
	7 $\vdash$ 9 & 0,22  \\
	9 $\vdash$ 11 & 0,26  \\
	11 $\vdash$ 13 & 0,20 \\
	13 $\vdash$ 15 & 0,14  \\
	\end{tabular}
	\\ Fonte: dados hipotéticos	
	\end{table}\\
discuta a aceitabilidade do estimador utilizado no dimensionamento da amostra e determine o correspondente intervalo de estimação. Na resolução do presente exercício, desenvolva desenvolva os cálculos, procedendo o arredondamento devido, até a segunda decimal.

\item Com a finalidade de estimar a proporção de unidades defeituosas em certa linha de fabricação, deve ser efetuado um levantamento por amostragem para o qual foi fixado um nível de confiança de 0,9544 e um erro absoluto de 0,04. Sabendo que, dada a inexistência de outras informações, foi efetuada uma pesquisa piloto sobre 40 unidades, verificando-se que 4 apresentaram defeito de fabricação, calcule o tamanho da amostra final. Sabendo, igualmente, que na amostra final levantada, 18 unidades apresentaram defeito de fabricação, estabeleça o intervalo de estimação desejado.

\item Sabendo que certo industrial deseja estimar a proporção de unidades produzidas em determinada linha de fabricação que apresentam defeito, fixado o nível de confiança de 0,9544 e um erro absoluto de 0,05, determine:
	\begin{enumerate}
	\item o tamanho da amostra;
	\item sabendo que na amostra levantada 24 unidades apresentaram defeito de fabricação, o intervalo de estimativa desejado.
	\end{enumerate}

\item Da população de certa região foi extraída uma amostra aleatória de 1.600 pessoas, na qual apurou-se que 1.280 eram torcedores do Sport Club Internacional. Considerando um nível de confiança de 0,9544, determine o intervalo de estimativa para a proporção de pessoas torcedoras daquela grande associação na referida região.

\item Da população de certo bairro foi extraída uma amostra aleatória de 100 moradores, verificando-se que 60 deles possuiam automóvel. Considerando um nível de confiança de 0,9544, determine o intervalo de estimativa para a proporção de moradores do referido bairro que possuem automóvel.

\item Com a finalidade de estimar a proporção de estabelecimentos produtores de determinada área geográfica, onde existem 2.000 unidades produtoras, que possuem equipamento mecanizado, deve ser efetuado um levantamento por amostragem para o qual foi fixado um nível de confiança de 0,9544 e um erro absoluto de 0,02. Considerando a inexistência de outras informações, foi executada uma pesquisa preliminar sobre 100 estabelecimentos, verificando-se que 75 possuíam equipamento mecanizado. Sabendo, finalmente, que na amostra final decorrente, 755 unidades produtoras apresentaram o atributo investigado, determine o intervalo de estimação desejado.

\item Com a finalidade de estimar a proporção de pessoas que possuem veículo automotor, em uma população constituída de 3.000 pessoas, deve ser efetuado um levantamento por amostragem para o qual foi estabelecido um nível de confiança de 0,9544 e um erro absoluto de 0,02. Sabendo que para cumprir tal objetivo, dada a inexistência de outras informações, foi efetuada uma pesquisa preliminar baseada em 40 unidades, tendo sido observado que 10 delas possuiam veículo automotor, calcule o tamanho da amostra final. Sabendo, finalmente, que nesta última pesquisa, 231 pessoas apresentaram o atributo estudado, determine o intervalo de estimação desejado.

\item Com a finalidade de estimar a proporção de famílias que possuem aparelho de televisão a cores, em certa cidade constituída de 2.000 famílias, deve ser efetuado um levantamento por amostragem para o qual foi estabelecido um nível de confiança de 0,9544 e um erro absoluto de 0,02. Sabendo que para cumprir tal objetivo, dada a inexistência de outras informações, foi efetuada uma pesquisa preliminar baseada em 40 unidades, tendo sido observado que 10 possuíam aparelho de televisão a cores, calcule o tamanho da amostra final. Sabendo, finalmente, que nesta última pesquisa, 242 famílias apresentaram o atributo estudado, determine o intervalo de estimação desejado.

\item Com a finalidade de estimar a média e o total da variável X, conforme cadastro $n^o$ 4, em anexo, deve ser efetuado um levantamento por amostragem para o qual foi estabelecido um nível de confiança de 0,9544 e um erro relativo de $10\%$.\\
Considerando a inexistência de informações sobre a variância absoluta e relativa da variável estudada, deve ser, inicialmente, realizada uma amostra piloto para a qual foi fixado um orçamento total de 1.400 unidades monetárias.\\
Sabe-se, por outro lado, que o custo fixo do planejamento de tal pesquisa preliminar é de 200 unidades monetárias e o custo médio por unidadevinvestigada é de 40 unidades monetárias.\\
Supondo, finalmente, que deva ser procedida nova seleção de unidades populacionais para os efeitos da estimação e considerando que o custo fixo de planejamento para essa segunda fase da pesquisa seja de 100 unidades monetárias e que o custo médio por unidade investigada não apresente qualquer alteração, determine o custo total da pesquisa.

\item Com a finalidade de estimar a média e o total da variável Y conforme cadastro $n^o$4, em anexo, deve ser efetuado um levantamento por amostragem para o qual foi estabelecido um nível de confiança de 0,9544 e um erro relativo de $10\%$.\\
Considerando a inexistência de informaçães sobre a variância absoluta e relativa da variável estudada, deve ser, inicialmente, realizada uma amostra piloto para a qual foi fixado um orçamento total de 2.200 unidades monetárias.\\
Sabe-se, por outro lado, que o custo fixo do planejamento de tal pesquisa preliminar é de 450 unidades monetárias e o custo médio por unidade investigada é de 50 unidades monetárias.\\
Supondo, finalmente, que deva ser procedida nova seleção de unidades populacionais para os efeitos da estimação e considerando que o custo fixo de planejamento para essa segunda fase da pesquisa seja de 400 unidades monetárias e que o custo médio por unidade investigada não apresente qualquer alteração, determine o custo total da pesquisa.

\item Com a finalidade de estimar a média e o total das variáveis X e Y, conforme cadastro $n^o$4, em anexo, deve ser efetuado um levantamento por amostragem para o qual foi estabelecido um nível de confiança de 0,9544 e um erro relativo de $10\%$.\\
Considerando a inexistência de informações sobre as variâncias absolutas e relativas das variáveis estudadas, deve ser, inicialmente, realizada uma amostra piloto para a qual foi fixado um orçamento total de 4.200 unidades monetárias.\\
Sabe- se, por outro lado, que o custo fixo do planejamento de tal pesquisa preliminar é de 1.600 unidades monetárias e o custo médio por unidade investigada é de 65 unidades monetárias.\\
Supondo, finalmente, que deva ser procedida nova seleção de unidades populacionais para os efeitos das estimaçoes e considerando que o custo fixo de planejamento para essa segunda fase da pesquisa seja de 1.500 unidades monetárias e que o custo médio por unidade investigada não apresente qualquer alteração, determine o custo total da pesquisa.

\item Com a finalidade de estimar a proporção de unidades que apresenta o atributo A, conforme cadastro $n^o$4, em anexo, deve ser efetuado um levantamento por amostragem para o qual foi fixado um nível de confiança de 0,9544 e um erro absoluto
de 0,1. Considerando a inexistência de outras informaçães, deve ser, inicialmente, realizada uma amostra piloto para a qual foi estabelecido um orçamento total de 5.500 unidades monetárias. Sabe- se, por outro lado, que o custo fixo do planejamento de tal pesquisa preliminar é de 2.300 unidades monetárias e o custo medio por unidade investigada é de 80 unidades monetárias.\\
Supondo, finalmente, que deva ser procedida nova seleção de unidades populacionais para os efeitos da estimação e considerando que o custo fixo do planejamento para essa segunda fase da pesquisa seja de 2.000 unidades monetárias e que o custo médio por unidades investigada não apresente qualquer alteração, determine o custo total da pesquisa.

\item Com a finalidade de estimar a média e o total das variáveis X e Y e a proporção de unidades que apresenta o atributo A, conforme cadastro $n^o$4, em anexo, deve ser efetuado um levantamento por amostragem para o qual foi estabelecido um nível de confiança de 0,9544, um erro relativo de $10\%$ para as variáveis e um erro absoluto de 0,1 para a proporção. Considerando a inexistência de outras informações, deve ser, inicialmente, realizada uma amostra piloto para a qual foi estabelecido um orçamento total de 8.000 unidades monetárias. Sabe-se, por outro lado, que o custo fixo de tal pesquisa preliminar é de 3.200 unidades monetárias e o custo médio por unidade investigada é de 120 unidades monetárias.\\
Supondo, finalmente, que deva ser procedida nova seleção de unidades populacionais para os efeitos da estimação e considerando que o custo fixo do planejamento para essa segunda fase da pesquisa seja de 2.500 unidades monetárias e que o custo médio por unidade investigada não apresente quaiquer alteração, determine o custo total da pesquisa.

\end{enumerate}