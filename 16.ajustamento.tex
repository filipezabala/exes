\chapter{Ajustamento}

\begin{enumerate}[resume]
%\setcounter{enumi}{1}

\item Considerando que a produção da Utilidade Z, dada em toneladas, apresenta o comportamento demonstrado na:
	\begin{table}[!htb]
	\centering
	\caption{Produção da Utilidade Z (t)}
	\vspace{0.5cm}
	\begin{tabular}{ll}
	Ano & Toneladas \\
	\hline 
	1969 & 56 \\
	1970 & 58 \\
	1971 & 64 \\
	1972 & 62 \\
	1973 & 68 \\
	\end{tabular}\\
	Fonte: dados hipotéticos
	\end{table}
	\\
	determine, utilizando o processo dos mínimos quadrados, a função linear ajustante, segundo o sistema das equações normais:
	\begin{enumerate}
	\item completas;
	\item simetrizadas, procedendo o correspondente retorno à variável de origem;
	\item a variância absoluta residual.	
	\end{enumerate}

\item Considerando que a produção da Utilidade Y, dada em toneladas, apresenta o comportamento demonstrado na:
	\begin{table}[!htb]
	\centering
	\caption{Produção da Utilidade Y (t)}
	\vspace{0.5cm}
	\begin{tabular}{ll}
	Ano & Toneladas \\
	\hline 
	1969 & 42,0 \\
	1970 & 43,5 \\
	1971 & 48,0 \\
	1972 & 46,5 \\
	1973 & 51,0 \\
	\end{tabular}\\
	Fonte: dados hipotéticos
	\end{table}
	\\
	determine, utilizando o processo dos mínimos quadrados, a função linear ajustante, segundo o sistema das equações normais:
	\begin{enumerate}
	\item completas;
	\item simetrizadas, procedendo a apresentação da correspondente função em termos não simetrizados de X;
	\item a variância absoluta residual.	
	\end{enumerate}
	
\item Considerando que a produção da Utilidade K, dada em toneladas, apresenta o comportamento demonstrado na:
	\begin{table}[!htb]
	\centering
	\caption{Produção da Utilidade K (t)}
	\vspace{0.5cm}
	\begin{tabular}{ll}
	Ano & Toneladas \\
	\hline 
	1973 & 38 \\
	1974 & 52 \\
	1975 & 60 \\
	1976 & 72 \\
	1977 & 78 \\
	\end{tabular}\\
	Fonte: dados hipotéticos
	\end{table}
	\\
	determine, utilizando o processo dos mínimos quadrados, a função linear e a parábola do 2$^o$ grau ajustantes, segundo o sistema das equações normais:
	\begin{enumerate}
	\item completas;
	\item simetrizadas, procedendo o correspondente retorno à variável de origem;
	\item a melhor função ajustante.
	\end{enumerate}	

\item Considerando que a produção da Utilidade W, dada em toneladas, apresenta o comportamento demonstrado na:
	\begin{table}[!htb]
	\centering
	\caption{Produção da Utilidade W (t)}
	\vspace{0.5cm}
	\begin{tabular}{ll}
	Ano & Toneladas \\
	\hline 
	1972 & 48 \\
	1973 & 60 \\
	1974 & 86 \\
	1975 & 106 \\
	1976 & 130 \\
	1977 & 170 \\
	\end{tabular}\\
	Fonte: dados hipotéticos
	\end{table}
	\\
	determine, utilizando o processo dos mínimos quadrados, a função linear e a parábola do 2$^o$ grau ajustantes, segundo o sistema das equações normais:
	\begin{enumerate}
	\item completas;
	\item simetrizadas, procedendo o correspondente retorno à variável de origem;
	\item a melhor função ajustante.
	\end{enumerate}		
	
\item Considerando que a produção da Utilidade A, dada em toneladas, apresenta o comportamento demonstrado na:
	\begin{table}[!htb]
	\centering
	\caption{Produção da Utilidade A (t)}
	\vspace{0.5cm}
	\begin{tabular}{ll}
	Ano & Toneladas \\
	\hline 
	1967 & 22 \\
	1968 & 28 \\
	1969 & 32 \\
	1970 & 44 \\	
	1971 & 56 \\	
	1972 & 66 \\
	1973 & 82 \\
	\end{tabular}\\
	Fonte: dados hipotéticos
	\end{table}
	\\
	determine, utilizando o processo dos mínimos quadrados, a parábola do 2$^o$ grau ajustante, de acordo com o sistema das equações normais:
	\begin{enumerate}
	\item completas;
	\item simetrizadas, procedendo o correspondente retorno à variável de origem;
	\item a variância absoluta residual.	
	\end{enumerate}			

\item Considerando que a produção da Utilidade B, dada em toneladas, apresenta o comportamento demonstrado na:
	\begin{table}[!htb]
	\centering
	\caption{Produção da Utilidade B (t)}
	\vspace{0.5cm}
	\begin{tabular}{ll}
	Ano & Toneladas \\
	\hline 
	1968 & 25 \\
	1969 & 30 \\
	1970 & 40 \\	
	1971 & 50 \\	
	1972 & 65 \\
	1973 & 70 \\
	\end{tabular}\\
	Fonte: dados hipotéticos
	\end{table}
	\\
	determine, utilizando o processo dos mínimos quadrados, a parábola do 2$^o$ grau ajustante, de acordo com o sistema das equações normais:
	\begin{enumerate}
	\item completas;
	\item simetrizadas, procedendo o correspondente retorno à variável de origem;
	\item a variância absoluta residual.	
	\end{enumerate}			

\item Dada a função ajustante $Yc_{i} = 125 + 5X'_{i}$, obtida por simetrização no período 1951/68 e representativa da produção anual da Utilidade K, dada em toneladas, determine a função ajustante em termos não simetrizados de X e indique o ano no qual a produção foi de 100 toneladas.

\item Dada a função ajustante $Yc_{i} = 124 + 2X'_{i}$, obtida por simetrização no período 1975/77 e representativa da produção mensal da Utilidade Y, dada em
toneladas, determine a função ajustante em termos não simetrizados de X e indique o mês no qual a produção foi de 118 toneladas.

\item  Dada a função ajustante $Yc_{i} = 150 + 24X'_{i}$,obtida por simetrização no período 1975/77 e representativa do volume físico das vendas quadrimestrais
de certo produto, determine a função ajustante em termos não simetrizados de X e indique o quadrimestre no qual foram vendidas 174 unidades.

\item  Dada a função ajustante $Yc_{i} = 50 + 5X_{i}$, correspondente à tonelagem anual produzida da Utilidade A, referente ao período 1958/68, determine a equação em termos simetrizados de X.

\item Dada a função ajustante $Yc_{i} = 100 + 20X_{i}$,  correspondente às vendas anuais de certo produto, referente ao período 1964/73, determine a equação em termos simetrizados de X.

\item Dada a função ajustante $Yc_{i} = 42,5 + 7X'_{i} + 0,5X'^{2}_{i}$, correspondente às vendas anuais de certo produto, referente ao período de 1965/73, determine a equação em termos não simetrizados de X e indique o ano no qual as vendas totalizaram 50 unidades monetárias.

\item Dada a função ajustante $Yc_{i} = 15,5 + 4X'_{i} + X'^{2}_{i} $, correspondente às vendas anuais de certo produto, calculada para o período 1966/73, determine a equação em termos não simetrizados de X e indique o ano no qual as vendas totalizaram 20,5 unidades monetárias.

\item Considerando que o volume físico das vendas trimestrais de certo produto seja dado pela função $Yc_{i} = 112,5 + 5X'_{i} + 0,5X'^{2}_{i}$, obtida por simetrização para o peíodo 1969/74, determine a equação em termos não simetrizados de X e indique o trimestre no qual foram vendidas 342 unidades.

\item Dada a função $Yc_{i} = 40 + 6X'_{i} + 8X'^{2}_{i}$, correspondente às vendas mensais de certo artigo, referente ao período 1974/77, determine a equação em termos simetrizados de X e indique o mês no qual as vendas totalizaram 8424 unidades monetárias.

\item Dada a função $Yc_{i} = 80 + 4X'_{i} + 2X'^{2}_{i}$, correspondente às vendas quadrimestrais de certo produto, referente ao período 1973/77, determine a equação em termos simetrizados de X e indique o quadrimestre no qual as vendas totalizaram 366 unidades monetárias.

\end{enumerate}