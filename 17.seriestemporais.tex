\chapter{Séries Temporais}

\begin{enumerate}[resume]
%\setcounter{enumi}{1}

\item Considerando que o volume físico das vendas da Utilidade Y apresenta o comportamento demonstrado na tabela que segue, proceda à análise da série correspondente e estabeleça a previsão, dada por intervalo, para os meses de janeiro, fevereiro e março de 1978:
	\begin{table}[!htb]
	\centering
	\caption{Produção da Utilidade Y}
	\vspace{0.5cm}
	\begin{tabular}{llll}
	Mês x Ano & 1975 & 1976 & 1977 \\
	\hline 
	jan. & 276 & 299 & 310 \\
	fev. & 266 & 288 & 322 \\
	mar. & 280 & 303 & 326 \\
	abr. & 281 & 304 & 327 \\
	mai. & 295 & 316 & 361 \\
	jun. & 312 & 329 & 385 \\	
	jul. & 300 & 329 & 385 \\
	ago. & 303 & 329 & 367 \\
	set. & 296 & 325 & 366 \\
	out. & 286 & 319 & 364 \\
	nov. & 287 & 307 & 339 \\
	dez. & 272 & 294 & 328 \\					
	\end{tabular}\\
	Fonte: dados hipotéticos
	\end{table}

\item Considerando que o volume físico das vendas da Utilidade A apresenta o comportamento demonstrado na tabela que segue, proceda à análise da série correspondente e estabeleça a previsão, dada por intervalo, para os meses de janeiro, fevereiro e março de 1978:
	\begin{table}[!htb]
	\centering
	\caption{Volume Físico das Vendas da Utilidade A}
	\vspace{0.5cm}
	\begin{tabular}{llll}
	Mês x Ano & 1975 & 1976 & 1977 \\
	\hline 
	jan. & 68 & 74 & 77 \\
	fev. & 66 & 71 & 79 \\
	mar. & 68 & 74 & 80 \\
	abr. & 70 & 76 & 82 \\
	mai. & 74 & 79 & 90 \\
	jun. & 78 & 82 & 98 \\	
	jul. & 76 & 84 & 98 \\
	ago. & 77 & 83 & 92 \\
	set. & 75 & 82 & 89 \\
	out. & 70 & 78 & 89 \\
	nov. & 72 & 77 & 85 \\
	dez. & 70 & 76 & 85 \\					
	\end{tabular}\\
	Fonte: dados hipotéticos
	\end{table}
	
\item Considerando que o volume físico das vendas da Utilidade B apresenta o comportamento demonstrado na tabela que segue, proceda à análise da série correspondente e estabeleça a previsão, dada por intervalo, para os trimestres de 1974:
	\begin{table}[!htb]
	\centering
	\caption{Volume Físico das Vendas da Utilidade B}
	\vspace{0.5cm}
	\begin{tabular}{lllll}
	Trimestre x Ano & 1970 & 1971 & 1972 & 1973 \\
	\hline 
	1$^o$ & 100 & 110 & 118 & 128 \\
	2$^o$ & 90 & 105 & 114 & 115 \\
	3$^o$ & 85 & 95 & 104 & 108 \\
	4$^o$ & 125 & 130 & 132 & 133 \\	
	\end{tabular}\\
	Fonte: dados hipotéticos
	\end{table}	
	
\item Considerando que a produção da Utilidade K apresenta o comportamento demonstrado na tabela que segue, proceda à análise da série correspondente e estabeleça a previsão, dada por intervalo, para os trimestres de 1974:
	\begin{table}[!htb]
	\centering
	\caption{Produção da Utilidade K (t)}
	\vspace{0.5cm}
	\begin{tabular}{llll}
	Trimestre x Ano & 1971 & 1972 & 1973 \\
	\hline 
	1$^o$ & 30 & 45 & 60 \\
	2$^o$ & 30 & 44 & 52 \\
	3$^o$ & 34 & 48 & 62 \\
	4$^o$ & 36 & 53 & 76 \\	
	\end{tabular}\\
	Fonte: dados hipotéticos
	\end{table}	
	
\item Considerando que a produção da Utilidade Y apresenta o comportamento demonstrado na tabela que segue, proceda à análise da série correspondente e estabeleça a previsão, dada por intervalo, para os trimestres de 1974:
	\begin{table}[!htb]
	\centering
	\caption{Produção da Utilidade Y (t)}
	\vspace{0.5cm}
	\begin{tabular}{llll}
	Trimestre x Ano & 1971 & 1972 & 1973 \\
	\hline 
	1$^o$ & 39 & 55 & 71 \\
	2$^o$ & 40 & 57 & 77 \\
	3$^o$ & 44 & 55 & 69 \\
	4$^o$ & 45 & 61 & 83 \\	
	\end{tabular}\\
	Fonte: dados hipotéticos
	\end{table}			
	
\item Considerando que o volume físico das vendas da Utilidade Y apresenta o comportamento demonstrado na tabela que segue, proceda à análise da série correspondente e estabeleça a previsão, dada por intervalo, para os semestres de 1979:
	\begin{table}[!htb]
	\centering
	\caption{Volume Físico das Vendas da Utilidade Y (t)}
	\vspace{0.5cm}
	\begin{tabular}{lllll}
	Semestre x Ano & 1975 & 1976 & 1977 & 1978 \\
	\hline 
	1$^o$ & 299 & 398 & 438 & 537 \\
	2$^o$ & 341 & 415 & 497 & 595 \\
	\hline 
	$\sum$ & 640 & 813 & 935 & 1132\\
	\end{tabular}\\
	Fonte: dados hipotéticos
	\end{table}		
	
\item Considerando que o volume físico das vendas trimestrais da Utilidade Y seja dado pela função $Yc_{i} = 46 + 2X'_{i}$, obtida por simetrização para o período 1971/73 e sabendo que $\sum (Y_{i} - Yc_{i} )^2 = 40,6272$ e que os respectivos coeficientes de estacionalidade correspondam aos demonstrados na:
	\begin{table}[!htb]
	\centering
 	\caption{Coeficientes de estacionalidade trimestrais do volume físico das vendas da Utilidade Y}
	\vspace{0.5cm}
	\begin{tabular}{ll}
	Trimestre & $S_{i}$ \\
	\hline 
	1$^o$ & 0,95 \\
	2$^o$ & 1,05 \\
	3$^o$ & 1,02 \\
	4$^o$ & 0,98 \\	
	\end{tabular}\\
	Fonte: dados hipotéticos
	\end{table}					
	\\
	determine a função ajustante em termos não simetrizados de X e estabeleça a previsão, dada por intervalo, para os trimestres de 1974.

\item Considerando que o volume físico das vendas trimestrais da Utilidade Y é dado pela função $Yc_{i} = 32,5 + 7X'_{i} + 0,5X'^2_{i}$, obtida por simetrização para os trimestres 1972/73 e sabendo que $\sum Y_{i} = 344$, $\sum (Y_{i} - Yc_{i} )^2 = 13,3128$ e que os respectivos coeficientes de estacionalidade correspondam aos demonstrados na:
	\begin{table}[!htb]
	\centering
 	\caption{Coeficientes de estacionalidade trimestrais do volume físico das vendas da Utilidade Y}
	\vspace{0.5cm}
	\begin{tabular}{ll}
	Trimestre & $S_{i}$ \\
	\hline 
	1$^o$ & 0,98 \\
	2$^o$ & 1,02 \\
	3$^o$ & 0,97 \\
	4$^o$ & 1,03 \\	
	\end{tabular}\\
	Fonte: dados hipotéticos
	\end{table}					
	\\
	determine a função ajustante em termos não simetrizados de X e estabeleça a previsão, dada por intervalo, para os trimestres de 1974.
	
\item Considerando as condições mencionadas no problema n$^{o}$ 202 e supondo a ausência da componente estacional, estabeleça a correspondente previsão, dada por intervalo, para os trimestres de 1974.

\item Considerando as condições mencionadas no problema n$^{o}$ 203 e supondo a ausência da componente estacional, estabeleça a correspondente previsão, dada por intervalo, para os trimestres de 1974.

\item Considerando que o volume físico das vendas trimestrais da Utilidade Y é de natureza estacionária e supondo a ausência da componente sazonal, admitindo que no período 1971/73 tenha sido observado que $\sum Y_{i}= 6.000$ e  $\sum (Y_{i}  - \mu _{y})^{2} = 7.500$, estabeleça a correspondente previsão, dada por intervalo, para os trimestres de 1974.

\item Considerando que o volume físico da produção mensal da Utilidade Y é de natureza estacionária e supondo a ausência da componente sazonal, admitido que no período 1972/73 tenha sido observado que $\sum Y_{i}= 4.800$ e $\sum (Y_{i}  - \mu _{y})^{2} = 1.536$, estabeleça a correspondente previsão, dada por intervalo, para os meses de 1974.

\item Considerando que o volume físico da produção trimestral da Utilidade Y e de natureza estacionária e supondo a presença da componente estacional, segundo os coeficientes constantes da Tabela n$^o$ 54, admitindo que no período 1972/73 tenha sido observado que $\sum Y_{i}= 2.400$ e $\sum (Y_{i}  - \mu _{y})^{2} = 1.152$, estabeleça a correspondente previsão, dada por intervalo, para os trimestres de 1974. 

\item Considerando que o volume físico da produção mensal da Utilidade Y e de natureza estacionária esupondo a presença da componente estacional, segundo os coeficientes apurados no problema n$^o$196, admitindo que no período 1970/73 tenha sido observado que $\sum Y_{i}= 7. 200$ e $\sum (Y_{i}  - \mu _{y})^{2} =  4.800$ , estabeleça a correspondente previsão, dada por intervalo, para os meses de janeiro, fevereiro e março de 1978.

\item Considerando que o volume fisico das vendas quadrimestrais de certo produto seja de natureza estacionária e não apresenta a componente sazonal, supondo que no período 1970/74 tenha sido observado que  $\sum Y_{i} = 3.000$ e $\sum (Y_{i}  - \mu _{y})^{2} =  960$, estabeleça a correspondente previsão, dada por intervalo, para os quadrimestres de 1975.

\item Considerando que o volume fisico das vendas anuais de certo produto seja dado pela função  $ Yc_{i} = 500 + 20 X'_{i} $ , obtida por simetrização para o período 1965/73 e sabendo que $\sum (Y_{i} - Yc_{i})^{2} = 900$, determine o ano no qual as vendas totalizaram 540 unidades e estabeleça o correspondente intervalo de estimação para o ano de 1974.

\item Considerando que o valor das vendas anuais de certo produto seja dado pela função $ Yc_{i} = 215 + 26 X'_{i} + (X'_{i})^{2}$, obtida por simetrização para o período 1966/75 e sabendo que $\sum Y_{i} =2.480$ e $\sum (Y_{i} - Yc_{i})^{2} =  984,064$, determine o ano no qual as vendas totalizaram 370 unidades monetárias e estabeleça o correspondente intervalo de estimação para o ano de 1976.


\end{enumerate}	