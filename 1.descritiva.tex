\chapter{Estatística Descritiva}

\begin{enumerate}

\item Considerando a renda anual da região ABC, segundo dados constantes do Cadastro n$^o$. 1 em anexo,
	\begin{enumerate}
	\item Construa a distribuição de frequências por classes de valores correspondente, determinando o n$^o$. de classes de mesma amplitude, mediante a utilização da formula de Sturges;
	\item Calcule os valores dos elementos característicos da distribuição decorrente;
	\item Interprete o significado dos seguintes elementos característicos:
		\begin{enumerate}
		\item frequência simples da 5$^a$ classe;
		\item frequência relativa da 5$^a$ classe;
		\item frequência acumulada da 5$^a$ classe;
		\item frequência relativa acumulada da 5$^a$ classe;
		\item ponto médio da 5$^a$ classe;
		\item limite inferior e superior da 5$^a$ classe;
		\end{enumerate}
	\item Construa os seguintes gráficos:
		\begin{enumerate}
		\item histograma de frequências simples;
		\item histograma de frequências relativas.
		\end{enumerate}
	\end{enumerate}

\item Considerando o número de dependentes por empregado da
empresa XYZ S/A, conforme dados constantes do Cadastro n$^o$ 2 em anexo,
	\begin{enumerate}
	\item Construa a distribuição de frequências por valores ou por ponto correspondente;
	\item Calcule os valores dos elementos característicos da distribuição decorrente;
	\item Interprete o significado dos seguintes elementos característicos:
		\begin{enumerate}
		\item frequência simples do 4$^o$ valor ou ponto;
		\item frequência acumulada do 5$^o$ valor ou ponto;
		\item frequência relativa do 5$^o$ valor ou ponto;
		\item frequência relativa acumulada do 3$^o$ valor ou ponto;
		\end{enumerate}
	\item Construa os seguintes gráficos:
		\begin{enumerate}
		\item em linha simples, utilizando as frequências simples;
		\item em linha simples, utilizando as frequências relativas;
		\item histograma de frequências simples;
		\item histograma de frequências relativas.
		\end{enumerate}
	\end{enumerate}
	
\item Considerando os salários pagos pela empresa ABC S/A, conforme dados constantes do Cadastro n$^o$ 3 em anexo, 
	\begin{enumerate}
	\item Construa a distribuição de frequências por classes de valores correspondente, adotando como limite inferior o valor zero e como superior o valor 14, fazendo a amplitude dos intervalos de classe constante e igual a duas unidades monetárias;
	\item Calcule os valores dos elementos característicos da distribuição decorrente;
	\item Interprete o significado dos seguintes elementos característicos:
		\begin{enumerate}
		\item frequência simples da segunda classe;
		\item frequência relativa da quinta classe;
		\item frequência acumulada da terceira classe;
		\item frequência relativa acumulada da quarta classe;
		\item ponto médio da sexta classe;
		\item limite inferior e superior da sexta classe;
		\end{enumerate}
	\item Construa os seguintes gráficos:
		\begin{enumerate}
		\item histograma de frequências simples;
		\item histograma de frequências relativas.
		\end{enumerate}
	\end{enumerate}

\item Considerando os dados constantes da:
	\begin{table}[!htb]
	\centering
	\caption{Produção da Utilidade A (t) no Brasil}
	\vspace{0.5cm}
	\begin{tabular}{ll}
	Mês & Toneladas \\
	\hline 
	janeiro & 20 \\
	fevereiro & 16 \\
	março & 22 \\
	abril & 18 \\
	maio & 22 \\
	junho & 24 \\
	julho & 28	 \\
	agosto & 26 \\
	setembro & 28 \\
	outubro & 30 \\
	novembro & 32 \\
	dezembro & 34 \\
	\end{tabular}
	 \newline \newline Fonte: dados hipotéticos
	\end{table}
	 determine o valor da: 
	\begin{enumerate}
		\item média aritmética, comprovando a propriedade que afirma que a soma algébrica dos desvios contados em relação à média aritmética é nula;
		\item variância absoluta, utilizando o processo de cálculo longo e abreviado;
		\item desvio padrão;
		\item variância relativa;
		\item coeficiente de variabilidade.
	\end{enumerate}
	
\item Considerando os dados da:
	\begin{table}[!htb]
	\centering
	\caption{Produção da Utilidade A (t) no Rio Grande do Sul}
	\vspace{0.5cm}
	\begin{tabular}{ll}
	Ano & Toneladas \\
	\hline 
	1967 & 6  \\
	1968 & 8  \\
	1969 & 7  \\
	1970 & 10  \\
	1971 & 9  \\
	1972 & 12  \\
	1973 & 11  \\
	\end{tabular}
	 \newline \newline Fonte: dados hipotéticos
	\end{table}
	determine o valor da:
	\begin{enumerate}
		\item média aritmética;
		\item variância absoluta, utilizando o processo de cálculo longo e abreviado;
		\item desvio padrão;
		\item variância relativa;
		\item coeficiente de variabilidade.
	\end{enumerate}

\item Considerando os dados da:
	\begin{table}[!htb]
	\centering
	\caption{Produção da Utilidade A (t) no Rio Grande do Sul}
	\vspace{0.5cm}
	\begin{tabular}{ll}
	Unidades Monetárias & N$^o$ de Pessoas \\
	\hline 
	0 $\vdash$ 2 & 9 \\
	2 $\vdash$ 4 & 13 \\
	4 $\vdash$ 6 & 16 \\
	6 $\vdash$ 8 & 45  \\
	8 $\vdash$ 10 & 56 \\
	10 $\vdash$ 12 & 35  \\
	12 $\vdash$ 14 & 14  \\
	14 $\vdash$ 16 & 9 \\
	16 $\vdash$ 18 & 3 \\ \\ \hline 
	$\sum$ & 200  \\ \hline 
	\end{tabular}
	 \newline \newline Fonte: dados hipotéticos
	\end{table}
	determine o valor da:
	\begin{enumerate}
		\item média aritmética, comprovando a propriedade que afirma que, o somatório dos produtos das frequências pelos desvios, contados em relação à média aritmética, é nulo;
		\item variância absoluta, utilizando o processo de cálculo longo e abreviado;
		\item desvio padrão;
		\item variância relativa;
		\item coeficiente de variabilidade.
	\end{enumerate}

\item Considerando os dados da:
	\begin{table}[!htb]
	\centering
	\caption{Dependentes por empregado da Empresa XYZ S/A  em Janeiro de 1974}
	\vspace{0.5cm}
	\begin{tabular}{ll}
	N$^o$ de Dependentes & N$^o$ de Empregados \\
	\hline 
	0 $\vdash$ 2 & 9 \\
	2 $\vdash$ 4 & 13 \\
	4 $\vdash$ 6 & 16 \\
	6 $\vdash$ 8 & 45  \\
	8 $\vdash$ 10 & 56 \\
	10 $\vdash$ 12 & 35  \\
	12 $\vdash$ 14 & 14  \\
	14 $\vdash$ 16 & 9 \\
	16 $\vdash$ 18 & 3 \\ \\ \hline 
	$\sum$ & 200  \\ \hline 
	\end{tabular}
	\newline Fonte: dados hipotéticos
	\end{table}
	determine o valor da:
	\begin{enumerate}
		\item média aritmética;
		\item variância absoluta, utilizando o processo de cálculo longo e abreviado;
		\item desvio padrão;
		\item variância relativa;
		\item coeficiente de variabilidade.
	\end{enumerate}

\item Considerando os dados da:
	\begin{table}[!htb]
	\centering
	\caption{Dependentes por empregado \\ Empresa XYZ S/A \\ Janeiro de 1974}
	\vspace{0.5cm}
	\begin{tabular}{ll}
	N$^o$ de Dependentes & N$^o$ de Empregados \\ \hline 
	0                    & 10                  \\
	1                    & 18                  \\
	2                    & 30                  \\
	3                    & 10                  \\
	4                    & 8                   \\
	5                    & 4                   \\ \hline 
	$\sum$  & 80  \\ \hline 
	\end{tabular}
	\newline \newline Fonte: dados hipotéticos
	\end{table}
	determine o valor da:
	\begin{enumerate}
		\item média aritmética;
		\item variância absoluta, utilizando o processo de cálculo longo e abreviado;
		\item desvio padrão;
		\item variância relativa;
		\item coeficiente de variabilidade.
	\end{enumerate}

\item Considerando os dados da:
	\begin{table}[!htb]
	\centering
	\caption{Salários pagos pela Empresa ABC S/A  \\ Janeiro de 1974}
	\vspace{0.5cm}
	\begin{tabular}{ll}
	\hline 
	Unidades Monetárias  & Empregados \\
	\hline 
	0 $\vdash$ 2 & 13 \\
	2 $\vdash$ 4 & 16 \\
	4 $\vdash$ 6 & 17 \\
	6 $\vdash$ 8 & 20  \\
	8 $\vdash$ 10 & 14 \\
	10 $\vdash$ 12 & 11 \\
	12 $\vdash$ 14 & 9  \\ \hline 
	$\sum$ & 100  \\ \hline 
	\end{tabular}
	\newline Fonte: dados hipotéticos
	\end{table}
	determine o valor da:
	\begin{enumerate}
		\item média aritmética;
		\item variância absoluta, utilizando o processo de cálculo longo e abreviado;
		\item desvio padrão;
		\item variância relativa;
		\item coeficiente de variabilidade.
	\end{enumerate}
	
\item Considerando que três distribuições hipotéticas apresentem os valores indicados na:
	\begin{table}[!htb]
	\centering
	\caption{Valores obtidos em três distribuições hipotéticas}
	\vspace{0.5cm}
	\begin{tabular}{lll}
	\hline 
	A  & B & C \\
	\hline 
	$ N = 200 $ & $ N = 50 $ &  $ \mu = 8 $ \\
	$ \sum f_{i}x_{i} = 4000$ & $ \sum f_{i}x_{i} = 500$ & $ \sum f_{i}x_{i} = 3200$ \\
	$ \sum f_{i}(x_{i}-\mu)^2 = 5000$ & $ \sum f_{i}x_{i}^2 = 5450$ & $ \sum f_{i}x_{i}^2 = 32000$ \\
	 \hline 
	\end{tabular}
	\newline Fonte: dados hipotéticos
	\end{table}
	\begin{enumerate}
		\item Determine os seguintes indicadores:
			\begin{table}[!htb]
			\begin{tabular}{llll}
			\hline 
			INDICADOR & A & B & C  \\
			\hline 
			Média Aritmética &     &     &      \\
			Variância  Absoluta &     &     &      \\
			Coef. de Variabilidade &     &     &      \\			 \hline 
			\end{tabular}
			\end{table}
		\item Baseado nos resultados acima obtidos, mencione a distribuição que apresenta maior
			\begin{enumerate}
				\item homogeneidade;
				\item heterogeneidade.
			\end{enumerate} 
		\end{enumerate} 

\item Considerando que quatro distribuições hipotéticas apresentem os valores indicados na:
	\begin{table}[!htb]
	\centering
	\caption{Valores obtidos em Quatro Distribuições Hipotéticas}
	\vspace{0.5cm}
	\begin{tabular}{llll}
	\hline 
	DISTRIBUIÇÃO & & & \\
	\hline 
	A  & B & C & D \\
	\hline 
	$ \mu = 8 $ & $ \mu = 100 $  & $N=100$ & $\mu = 50$  \\
	$\sigma^2=4$ & $\sigma^2=121$ &  $\sum f_{i}x_{i}= 5000$ &  $\sum f_{i}x_{i}= 10000$  \\
	    &    &  $\sum f_{i}x_{i}^2= 256400$ &  $\sum f_{i}(x_{i}-\mu)^2= 7200$  \\
	\hline 
	\end{tabular}
	\newline Fonte: dados hipotéticos
	\end{table}
	Indique a distribuição que apresenta:
	\begin{enumerate}
	\item menor dispersão relativa em torno da média aritmética;
	\item maior dispersão relativa em torno da média aritmética.
	\end{enumerate}	
	
\item Considerando que quatro distribuições hipotéticas apresentem os valores indicados na:
	\begin{table}[!htb]
	\centering
	\caption{Valores obtidos em Quatro Distribuições Hipotéticas}
	\vspace{0.5cm}
	\begin{tabular}{llll}
	\hline 
	DISTRIBUIÇÃO & & & \\
	\hline 
	A  & B & C & D \\
	\hline 
	    & $N=100$  & $\mu=50$ & $N=200$  \\
	$\gamma^2=0,0169$ & $\sum f_{i}x_{i}=2000$ &   &  $\sum f_{i}x_{i}= 8000$  \\
	    &  $\sum f_{i}x_{i}^2=42500$   &  $\sigma^2=625$ &  $\sum f_{i}(x_{i}-\mu)^2= 3200$  \\
	\hline 
	\end{tabular}
	\newline Fonte: dados hipotéticos
	\end{table}
	Indique a distribuição que apresenta:
	\begin{enumerate}
	\item Maior homogeneidade;
	\item Maior heterogeneidade.
	\end{enumerate}

\item Sabendo que a variável X, que assume as determinações X$_{i}$ (i = 1, 2, 3, ...,N), possui $\mu _{x} = 10$ e $(\sigma _{x})^2 =16$, determine a média aritmética e a variância absoluta da variável $Y =X + 2$.

\item Sabendo que a variável X,que assume as determinações X$_{i}$ (i = 1, 2 ,3,...N), possui $\mu _{x} = 10$ e $(\sigma _{x})^2 =16$ determine a média aritmética e a variância absoluta da variável$Y= 2X$.

\item Sabendo que a variável X,que assume as determinações X$_{i}$ (i = 1, 2 ,3,...N), possui $\mu _{x} = 10$ e $(\sigma _{x})^2 =16$ determine a média aritmética e a variância absoluta da variável $Y= \frac{X}{2} + 4$.

\item Sabendo que a variável X,que assume as determinações X$_{i}$ (i = 1, 2 ,3,...N), possui $\mu _{x} = 10$ e $(\sigma _{x})^2 =16$ determine a média aritmética e a variância absoluta da variável $Y= \frac{X}{2} - 4$.

\item Sabendo que a variável X,que assume as determinações X$_{i}$ (i = 1, 2 ,3,...N), possui $\mu _{x} = 6$ e $\gamma ^2 =0,25$, determine a média aritmética e a variância absoluta da variável $Y= \frac{X}{4} + 2$.



\end{enumerate}  