\chapter{Distribuição Binomial}

\begin{enumerate}[resume]
%\setcounter{enumi}{1}

\item A probabilidade de exemplar defeituoso com que opera certo processo produtivo é de 0, 10. Considerando X a variável número de unidades defeituosas em uma amostra ocasional de quatro unidades, determine:
	\begin{enumerate}
	\item a distribuição de probabilidade associada a X;
	\item	a expectância, a variância absoluta e o coeficiente de variabilidade correspondente;
	\item a $P (X \leq 3)$;
	\item a $P(X > 2)$; 
	\item a $P(1 < X \leq 3)$;
	\item	a $P(1 \leq X < 3)$.	
	\end{enumerate}

\item Sendo X a variável aleatória número de vezes que ocorre a face cara no triplo lançamento de uma moeda perfeita ou no lançamento simultâneo de três moedas em iguais condições, determine:
	\begin{enumerate}
	\item a distribuição de probabilidade associada a X;
	\item	a expectância, a variância absoluta e o coeficiente de variabilidade correspondente;
	\item a $P (X > 2)$;
	\item a $P(X \leq  3)$; 
	\item a $P(0 < X \leq 2)$;
	\item	a $P(1 \leq X < 3)$.	
	\end{enumerate}

\item Sendo X a variável aleatória número de vezes que ocorre o evento número par no triplo lançamento de um dado perfeito ou no lançamento simultâneo de três dados em iguais condições, determine:
	\begin{enumerate}
	\item a distribuição de probabilidade associada a X;
	\item	a expectância, a variância absoluta e o coeficiente de variabilidade correspondente;
	\item a $P (X \leq 2)$;
	\item a $P(X > 1)$; 
	\item a $P(0 < X \leq 2)$;
	\item	a $P(1 \leq X < 3)$.	
	\end{enumerate}
	
\item Sendo X a variável aleatória número de vezes que ocorre o evento número múltiplo de dois ou de três no triplo lançamento de um dado perfeito ou no lançamento simultâneo de três dados em iguais condições, determine:
	\begin{enumerate}
	\item a distribuição de probabilidade associada a X;
	\item	a expectância, a variância absoluta e o coeficiente de variabilidade correspondente;
	\item a $P (X \leq 1)$;
	\item a $P(X > 1)$; 
	\item a $P(0 < X \leq 2)$;
	\item	a $P(1 \leq X < 3)$.	
	\end{enumerate}

\item A probabilidade de exemplar defeituoso com que opera certo processo produtivo e de 0,20. Considerando X a variável aleatória número de unidades defeituosas em uma amostra ocasional de cinco unidades, determine:
	\begin{enumerate}
	\item a distribuição de probabilidade associada a X;
	\item	a expectância, a variância absoluta e o coeficiente de variabilidade correspondente;
	\item a $P (X \leq 4)$;
	\item a $P(X > 2)$; 
	\item a $P(1 < X \leq 4)$;
	\item	a $P(2 \leq X < 4)$.	
	\end{enumerate}

\item Sabendo que certo processo industrial produz, em média, $20\%$ de unidades defeituosas e considerando uma amostra ocasional de quatro unidades, determine a probabilidade de se obter:
	\begin{enumerate}
	\item nenhuma unidade defeituosa;
	\item uma unidade defeituosa;
	\item mais de uma unidade defeituosa;
	\item ao menos uma unidade defeituosa;
	\item um número de unidades defeituosas no intervalo $[0;1]$.
	\end{enumerate}

\item Sabendo que certo processo industrial produz, em média, $10\%$ de unidades defeituosas e considerando uma amostra ocasional de cinco unidades, determine a probabilidade de se obter:
	\begin{enumerate}
	\item nenhuma unidade defeituosa;
	\item uma unidade defeituosa;
	\item mais de uma unidade defeituosa;
	\item ao menos uma unidade defeituosa;
	\item um número de unidades defeituosas no intervalo $[0;1]$.
	\end{enumerate}
	
\item Sabendo que a probabilidade de um estudante obter aprovação em certo teste de Estatística é igual a 0,4 e considerando um grupo de cinco estudantes, determine a probabilidade de que: 
	\begin{enumerate}
	\item nenhumseja aprovado;
	\item apenas um seja aprovado;
	\item ao menos um seja aprovado;
	\item dois sejam aprovados;
	\item no máximo dois sejam aprovados.
	\end{enumerate}
	
\end{enumerate}