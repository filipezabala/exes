\chapter{Deflacionamento}

\begin{enumerate}[resume]
%\setcounter{enumi}{1}

\item Considerando a série de relativos de base fixadas vendas totais da Utilidade A e o índice de preços ao consumidor, calculados para determinada área geográfica, conforme dados da tabela abaixo, determine:
	\begin{enumerate}
	\item a série de valores das vendas totais a preços correntes;
	\item a série de valores das vendas totais a preços constantes.
	\end{enumerate}
	\begin{table}[!htb]
	\centering
	\caption{Relativos das Vendas Totais da Utilidade A da Região ABC}
	\vspace{0.5cm}
	\begin{tabular}{lll}
	Ano & Relativos & IPC \\
	\hline 
	1969 & 64 & 1,00  \\
	1970 & 80 & 1,24  \\
	1971 & 100 & 1,53  \\
	1972 & 120 & 1,81  \\
	1973 & 132 & 1,96  \\
	\end{tabular}\\
	Fonte: dados hipotéticos\\
	Base: 400 unidades monetárias
	\end{table}
	
\item Considerando a série de relativos de base fixadas vendas totais da Utilidade B e o índice de preços ao consumidor, calculados para determinada área geográfica, conforme dados da tabela abaixo, determine a correspondente série:
	\begin{enumerate}
	\item de valores das vendas totais a preços correntes;
	\item de valores das vendas totais a preços constantes.
	\end{enumerate}
	\begin{table}[!htb]
	\centering
	\caption{Relativos das Vendas Totais da Utilidade B da Região ABC}
	\vspace{0.5cm}
	\begin{tabular}{lll}
	Ano & Relativos & IPC \\
	\hline 
	1968 & 80,0 & 1,00  \\
	1969 & 100,0 & 1,20  \\
	1970 & 120,0 & 1,25  \\
	1971 & 138,0 & 1,38  \\
	1972 & 165,6 & 1,50  \\
	\end{tabular}\\
	Fonte: dados hipotéticos\\
	Base: 600 unidades monetárias
	\end{table}	
	
\item Considerando os dados da tabela abaixo, verifique a variação real dos salários da empresa ABC S/A e interprete os resultados obtidos:
	\begin{table}[!htb]
	\centering
	\caption{Salários Pagos por ABC S/A (R\$)}
	\vspace{0.5cm}
	\begin{tabular}{lll}
	Ano & Salários pagos & IPC \\
	\hline 
	1968 & 21 & 0,70  \\
	1969 & 34 & 0,80  \\
	1970 & 45 & 0,90  \\
	1971 & 55 & 1,00  \\
	1972 & 66 & 1,10  \\
	1973 & 78 & 1,20  \\	
	\end{tabular}\\
	Fonte: dados hipotéticos
	\end{table}		
	
\item Considerando a série de relativos de base fixa das vendas totais da Utilidade K e o índice de preços ao consumidor, conforme os dados da tabela abaixo, apure a correspondente série dos valores das vendas totais a preços constantes e interprete o valor obtido para o ano de 1973:
	\begin{table}[!htb]
	\centering
	\caption{Relativos das Vendas da Utilidade K}
	\vspace{0.5cm}
	\begin{tabular}{lll}
	Ano & Relativos & IPC \\
	\hline 
	1969 & 100,0 & 1,00  \\
	1970 & 123,2 & 1,10  \\
	1971 & 133,4 & 1,15  \\
	1972 & 144,0 & 1,20  \\
	1973 & 155,0 & 1,25  \\	
	\end{tabular}\\
	Fonte: dados hipotéticos\\
	Base: 500 unidades monetárias
	\end{table}			
	
\item Considerando a série de relativos de base fixa do volume financeiro das vendas do produto Z e o índice de preços ao consumidor, conforme os dados da:
	\begin{table}[!htb]
	\centering
	\caption{Relativos das Vendas do Produto Z da Região ABC}
	\vspace{0.5cm}
	\begin{tabular}{lll}
	Ano & Relativos & IPC \\
	\hline 
	1972 & 100,0 & 1,00  \\
	1973 & 126,0 & 1,12  \\
	1974 & 142,6 & 1,15  \\
	1975 & 154,2 & 1,20  \\			
	1976 & 165,8 & 1,25  \\
	\end{tabular}\\
	Fonte: dados hipotéticos\\
	Base: R\$ 15.000,00
	\end{table}	
	\\
	determine:	
	\begin{enumerate}
	\item o valor das vendas a preços correntes;
	\item o valor real das vendas, fixado o poder aquisitivo da moeda de 1972;
	\item a variação real das vendas, em termos absolutos e percentuais, do ano de 1976 em relação ao de 1972, mantido constante o poder aquisitivo da moeda de 1972;
	\item o valor real das vendas, fixado o poder aquisitivo da moeda de 1976;
	\item a variação real das vendas, em termos absolutos e percentuais, do ano de 1976 em relação ao de 1972, mantido constante o poder aquisitivo da moeda de 1976.
	\end{enumerate}

\item Considerando a série de relativos de base fixa do volume financeiro das vendas totais da utilidade Y e o índice de preços ao consumidor, conforme os dados da:
	\begin{table}[!htb]
	\centering
	\caption{Relativos das Vendas da Utilidade Y}
	\vspace{0.5cm}
	\begin{tabular}{lll}
	Ano & Relativos & IPC \\
	\hline 
	1970 & 85,4 & 0,98  \\
	1971 & 92,8 & 1,00  \\
	1972 & 102,0 & 1,08  \\
	1973 & 105,2 & 1,10  \\
	1974 & 110,5 & 1,15  \\
	1975 & 121,8 & 1,25  \\			
	\end{tabular}\\
	Fonte: dados hipotéticos\\
	Base: valor médio das vendas do triênio 1971/73, R\$ 12.000,00
	\end{table}		
	\\
	determine:
	\begin{enumerate}
	\item o valor das vendas do ano de 1975 a preços correntes;
	\item fixado o poder aquisitivo de 1971, o valor real das vendas do ano de 1975;
	\item fixado o poder aquisitivo de 1970, o valor real das vendas do ano de 1975;
	\item fixado o poder aquisitivo de 1975, o valor real das vendas do ano de 1973.
	\end{enumerate}	

\item Considerando a série de relativos de base fixa do volume financeiro das vendas totais da utilidade Y e o índice de preços ao consumidor, com base no ano de 1968, conforme os dados da:
	\begin{table}[!htb]
	\centering
	\caption{Relativos das Vendas da Utilidade Y}
	\vspace{0.5cm}
	\begin{tabular}{lll}
	Ano & Relativos & IPC \\
	\hline 
	1972 & 71,5 & 1,10  \\
	1973 & 75,9 & 1,15  \\
	1974 & 93,0 & 1,20  \\
	1975 & 107,0 & 1,25  \\	
	1976 & 112,0 & 1,28  \\
	1977 & 122,2 & 1,30  \\			
	\end{tabular}\\
	Fonte: dados hipotéticos\\
	Base: média das vendas do período 1974/1975, R\$ 10.000,00
	\end{table}	
	\\
	determine:	
	\begin{enumerate}
	\item o valor das vendas a preços correntes;
	\item o valor real das vendas, fixado o poder aquisitivo da moeda de 1968;
	\item a variação real das vendas, em termos absolutos e percentuais, do ano de 1977 em relação ao de 1972, mantido constante o poder aquisitivo da moeda de 1968;
	\item o valor real das vendas, fixado o poder aquisitivo da moeda de 1975;
	\item a variação real das vendas, em termos absolutos e percentuais, do ano de 1977 em relação ao de 1972, mantido constante o poder aquisitivo da moeda de 1975.
	\end{enumerate}	


\end{enumerate}