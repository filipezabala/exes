\chapter{Pequenas Amostras}

\begin{enumerate}[resume]
%\setcounter{enumi}{1}

\item A Industrial XYZ S/A, fabricante de lâmpadas elétricas, desejando conhecer o tempo médio de duração de seu produto, selecionou uma amostra aleatória de 25 unidades, apurando sua média em 250 horas e a sua variância absoluta em 384. Fixado o nível de confiança de 0,95, estabeleça o correspondente intervalo de estimação.

\item A Industrial ABC S/A, desejando conhecer o comprimento médio, dado em metros, das barras de ferro que produz, selecionou uma amostra aleatória de 17 unidades, apurando sua média em 10 metros e seu desvio padrão em 0,80 metros. Fixado o nível de confiança de 0,90, estabeleça o correspondente intervalo de estimação.

\item Um fabricante de tijolos, desejando conhecer a resistência ao rompimento de seu produto selecionou uma amostra aleatória de 26 unidades, apurando sua média em 200 quilos e o seu desvio padrão em 10 quilos. Fixado o nível de confiança de 0,95, estabeleça o correspondente intervalo de estimaçao.

\item A Industrial ABC S/A, fabricante de lâmpadas elétricas, desejando conhecer o tempo médio de duração do seu produto, selecionou uma amostra aleatória de 10 unidades, apurando que  $\sum X_{i} = 2.740$ e  $\sum X_{i}^{2} = 751.120$. Fixado o nivel de confiança de 0,95, estabeleça o correspondente intervalo de estimação.

\item Com a finalidade de estimar a média paramétrica da variável X, que representa determinada característica de uma população infinita, foi extraída uma amostra aleatória constituída de 26 unidades, verificando-se que $\sum X_{i} = 520$ e $\sum X_{i}^{2} = 11.050$. Fixado o nível de confiança de 0,95, estabeleça o correspondente intervalo de estimação.

\item Com a finalidade de estimar a média paramétrica da variável X, que representa determinada característica de uma população infinita, foi extraída uma amostra ocasional constituída de 17 unidades, verificando-se que $\sum X_{i} = 850$ e $\sum X_{i}^{2} = 42. 772$. Fixado o nível de confiança de 0,95, estabeleça o correspondente intervalo de  estimação.

\item Numa amostra de 10 lâmpadas elétricas produzidas por uma empresa, verificou-se que o seu tempo médio de duração foi calculado em 490 horas e seu desvio padrão em 12 horas. Fixado o nível de significância de 0,05, testar a hipótese da vida média de todas as lâmpadas produzidas por tal empresa ser de 500 horas contra a hipótese alternativa de sua média ser diferente de 500 horas.

\item Um fabricante de conservas anuncia que o conteúdo líquido de uma lata de seu produto é de 900 gramas. A fiscalização de pesos e medidas investigou uma amostra aleatória de 17 latas, apurando sua média em 890 gramas e seu desvio padrão em 20 gramas. Fixado o nível de significância de 0,05, deverá o fabricante ser multado por não efetuar a venda do produto conforme anuncia ?

\item Um fabricante produz determinado artigo, cujo tempo medio de duração é de 200 horas. Com o objetivo de aumentar o tempo médio de duração de seu produto, foi introduzido um novo esquema de fabricação. Para examinar a durabilidade dos artigos produzidos pelo novo processo, foram testadas 10 unidades, apurando-se $\bar{X} = 207,6$ e $s = 12$. Fixado o nivel de significância de 0,05, verifique se o novo processo produz o artigo desejado?

\item Certo fabricante de alimentos em conserva supõe que a máquina de enlatar seu produto, face a problemas de ordem técnica, esteja colocando, em os da unidade, um peso maior do que o especificado que é de 200 gramas. Para verificar a veracidade
de tal suposição, foi investigada uma amostra de 17 unidades, constatando- se que $\sum X_{i} = 3.417$ e $\sum X_{i}^{2} = 686. 885$. Fixado o nível de significância de 0,05, estabeleça qual a conclusão do referido industrial.

\item A Industrial ABC S/A, fabricante de determinado equipamento eletrônico, procedeu à substituição de certo componente importado pelo similar nacional. Um grande comprador da referida indústria supõe que tal substituição tenha diminuído a duração do produto que antes era anunciada como sendo, em média, de duzentas horas. Para julgar da aceitabilidade de sua suposição, o comprador testou uma amostra de 10 unidades, verificando que $\sum X_{i} = 1.970$ e $\sum X_{i}^{2} = 388.450$. Fixado o nível de significância de 0,05, estabeleça a conclusão alcançada pelo comprador.

\item Um fabricante produz determinado artigo cujo .tempo médio de duração é de 25,8 horas. Com o objetivo de aumentar o tempo médio de duração, foi introduzido um novo esquema de fabricação. Para examinar a durabilidade dos artigos produzidos pelo novo processo, foram testadas cinco unidades apurando-se que\\
	\begin{table}[!htb]
	\centering
	\begin{tabular}{ll}
	Unidade & Duração (horas) \\
	\hline 
	$U_{1}$ & 30  \\
	$U_{2}$ & 36 \\
	$U_{3}$ & 32 \\
	$U_{4}$ & 28  \\
	$U_{5}$ & 24  \\
	\end{tabular}
	\end{table}\\
Fixado o nível de significância de 0,05 verifique se o novo processo produz o artigo desejado.

\item Certo fabricante afirmava que o tempo médio de duração de seu produto era de 12,5 horas. Face a utilização da matéria-prima de qualidade inferior, por limitações de mercado, o fabricante supõe que a duração média de seu produto tenha sido alterada. Para examinar a durabilidade dos artigos produzidos sob as novas condiçoes, foram testadas cinco unidades apurando-se que\\
	\begin{table}[!htb]
	\centering
	\begin{tabular}{ll}
	Unidade & Duração (horas) \\
	\hline 
	$U_{1}$ & 10  \\
	$U_{2}$ & 7 \\
	$U_{3}$ & 9 \\
	$U_{4}$ & 13  \\
	$U_{5}$ & 11  \\
	\end{tabular}
	\end{table}\\
Fixado o nível de significância de 0,05 verifique sa conclusão obtida pelo produtor. 

\end{enumerate}