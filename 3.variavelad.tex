\chapter{Variável Aleatória Discreta}

\begin{enumerate}[resume]
%\setcounter{enumi}{1}

\item Seja uma urna que contém quatro fichas azuis, seis fichas brancas e dez fichas cinzas. Considerando X a variável aleatória que assume o valor zero, um e dois quando ocorre, respectivamente, a extração de ficha azul, branca e cinza, determine: 
	\begin{enumerate}
	\item a distribuição de probabilidade associada a X;
	\item a expectância, a variância absoluta e o coeficiente de variabilidade correspondente;
	\item a $P(X \leq 1)$;
	\item a $P (X > 0)$;
	\item  a $P(0 < X \leq 2)$.
	\end{enumerate}

\item Seja uma urna que contém duas fichas azuis, quatro fichas brancas, seis fichas cinzas e oito fichas verdes. Considerando X a variável aleatória que assume o valor zero, um, dois e três quando ocorre, respectivamente, a extração de ficha azul, branca, cinza e verde, determine:
	\begin{enumerate}
	\item a distribuição de probabilidade associada a X;
	\item a expectância, a variância absoluta e o coeficiente de variabilidade correspondente;
	\item a $P(X \leq 1)$;
	\item a $P (X > 1)$;
	\item  a $P(1 < X \leq 3)$.
	\end{enumerate}

\item Seja uma urna que contém três fichas azuis, cinco fichas brancas e duas fichas cinzas. Considerando X a variável aleatória que assume o valor zero, um e dois quando ocorre, respectivamente, a extração de uma ficha azul, branca e cinza, determine:
	\begin{enumerate}
	\item a distribuição de probabilidade associada a X;
	\item a expectância e o coeficiente de variabilidade correspondentes;
	\item a $P(X \leq 1)$;
	\item a $P (X > 0)$;
	\item  a $P(0 < X \leq 2)$.
	\end{enumerate}
	
\item Considerando a distribuição de probabilidade da variável aleatória X, constante da seguinte:
	\begin{table}[!htb]
	\centering
	\caption{Distribuição de Probabilidade da Variável Aleatória X}
	\vspace{0.5cm}
	\begin{tabular}{ll}
	X & f(X) \\
	\hline 
	0 & 0,15 \\
	1 & 0,20 \\
	2 & 0,35 \\
	3 & 0,15 \\
	4 & 0,10 \\
	5 & 0,05 \\
	\hline
	$\sum$ & 1,00
	\end{tabular}
	 \newline \newline Fonte: dados hipotéticos
	\end{table}
	determine:
	\begin{enumerate}
	\item a expectância, a variância absoluta e o coeficiente de variabilidade correspondente;
	\item a $P(X \leq 3)$;
	\item a $P(X > 2)$;
	\item a $P(1 < X \leq 4)$;
	\item a $P(0 < X \leq 5)$.			
	\end{enumerate}
	
\item Considerando a distribuição de probabilidade da variável aleatória X, constante da seguinte:
	\begin{table}[!htb]
	\centering
	\caption{Distribuição de Probabilidade da Variável Aleatória X}
	\vspace{0.5cm}
	\begin{tabular}{ll}
	X & f(X) \\
	\hline 
	0 & 0,05 \\
	1 & 0,30 \\
	2 & 0,35 \\
	3 & 0,20 \\
	4 & 0,10 \\
	\hline
	$\sum$ & 1,00
	\end{tabular}
	 \newline \newline Fonte: dados hipotéticos
	\end{table}
	determine:
	\begin{enumerate}
	\item a expectância, a variância absoluta e o coeficiente de variabilidade correspondente;
	\item a $P(X \leq 1)$;
	\item a $P(X > 2)$;
	\item a $P(1 < 3)$;
	\item a $P(1 < X \leq 3)$.			
	\end{enumerate}	
	
\end{enumerate}  