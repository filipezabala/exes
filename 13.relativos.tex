\chapter{Relativos}

\begin{enumerate}[resume]
%\setcounter{enumi}{1}

\item Considerando os dados constantes da tabela abaixo, determine os relativos de base móvel e, fixada a base em 1963, os correspondentes relativos de base fixa.
	\begin{table}[!htb]
	\centering
	\caption{Produção da Utilidade A (toneladas)}
	\begin{tabular}{ll}
	Ano & Toneladas \\
	\hline 
	1960 & 160  \\
	1961 & 190  \\
	1962 & 180  \\
	1963 & 200  \\
	1964 & 220  \\
	1965 & 230  \\
	1966 & 240  \\
	\end{tabular} 
	\\ Fonte: dados hipotéticos
	\end{table}

\item Considerando os relativos de base fixa constantes na tabela abaixo, calcule as correspondentes determinaçães da variável Produção da Utilidade B.
	\begin{table}[!htb]
	\centering
	\caption{Relativos Produção da Utilidade B (toneladas)}
	\begin{tabular}{ll}
	Ano & Relativos \\
	\hline 
	1960 & 75  \\
	1961 & 90  \\
	1962 & 100  \\
	1963 & 105  \\
	1964 & 110  \\
	1965 & 120  \\
	1966 & 125  \\
	\end{tabular} 
	\\ Fonte: dados hipotéticos \\
	Base: 500 toneladas	
	\end{table}
	
\item Considerando os dados constantes da tabela abaixo, determine os relativos de base móvel e, fixada a base em 1970, os correspondentes relativos de base fixa.
	\begin{table}[!htb]
	\centering
	\caption{Preços da Utilidade C (R\$)}
	\begin{tabular}{ll}
	Ano & R\$ \\
	\hline 
	1968 & 25  \\
	1969 & 40  \\
	1970 & 50  \\
	1971 & 60  \\
	1972 & 75  \\
	1973 & 90  \\
	\end{tabular} 
	\\ Fonte: dados hipotéticos
	\end{table}
	
\item Considerando os relativos de base fixa constantes da tabela abaixo, determine os correspondentes relativos de base móvel e indique o total da produção no ano de 1971.
	\begin{table}[!htb]
	\centering
	\caption{Relativos da Produção da Utilidade Y}
	\begin{tabular}{ll}
	Ano & Relativos \\
	\hline 
	1967 & 40  \\
	1968 & 50  \\
	1969 & 100  \\
	1970 & 135  \\
	1971 & 162  \\
	\end{tabular} 
	\\ Fonte: dados hipotéticos\\
	Base: 300 toneladas
	\end{table}
	
\item Considerando os relativos de base fixa constantes da tabela abaixo, determine os correspondentes relativos de base móvel e indique o preço referente ao ano de 1973. 
	\begin{table}[!htb]
	\centering
	\caption{Relativos da Produção da Utilidade X}
	\begin{tabular}{ll}
	Ano & Relativos \\
	\hline 
	1968 & 50,0 \\
	1969 & 80,0 \\
	1970 & 100,0 \\
	1971 & 128,0 \\
	1972 & 140,8  \\
	1973 & 211,2  \\
	\end{tabular} 
	\\ Fonte: dados hipotéticos\\
	Base: R\$ 120,0
	\end{table}		

\item Considerando os relativos de base móvel constantes da tabela abaixo, determine os correspondentes relativos de base fixa.
	\begin{table}[!htb]
	\centering
	\caption{Relativos de Base Móvel dos Preços da Utilidade K}
	\begin{tabular}{ll}
	Ano & Relativos \\
	\hline 
	1967 & 120  \\
	1968 & 130  \\
	1969 & 150  \\
	1970 & 125  \\
	1971 & 160  \\
	1972 & 110  \\
	1973 & 175  \\		
	\end{tabular} 
	\\ Fonte: dados hipotéticos
	\end{table}
	
\item Considerando os relativos de base móvel constantes da tabela abaixo, determine os correspondentes relativos de base fixa.	
	\begin{table}[!htb]
	\centering
	\caption{Relativos de Base Móvel dos Preços da Utilidade Z}
	\begin{tabular}{ll}
	Ano & Relativos \\
	\hline 
	1968 & 130  \\
	1969 & 150  \\
	1970 & 120  \\
	1971 & 150  \\
	1972 & 110  \\
	1973 & 150  \\		
	\end{tabular} 
	\\ Fonte: dados hipotéticos
	\end{table}
	
\item Considerando os relativos de base fixa constantes da tabela abaixo, proceda à mudança da base para o ano de 1971 e determine o total da produção para o triênio 1971/73.
	\begin{table}[!htb]
	\centering
	\caption{Relativos de Base Móvel dos Preços da Utilidade K}
	\begin{tabular}{ll}
	Ano & Relativos \\
	\hline 
	1968 & 100  \\
	1969 & 120  \\
	1970 & 130  \\
	1971 & 125  \\
	1972 & 140  \\
	1973 & 150  \\		
	\end{tabular} 
	\\ Fonte: dados hipotéticos\\
	Base: 200 toneladas 
	\end{table}
	
\item Considerando os relativos de base fixa constantes da tabela abaixo, proceda à mudança da base para o ano de 1972 e determine o os preços correspondentes aos anos 1970/73.
	\begin{table}[!htb]
	\centering
	\caption{Relativos de Base Móvel dos Preços da Utilidade J}
	\begin{tabular}{ll}
	Ano & Relativos \\
	\hline 
	1968 & 80  \\
	1969 & 88  \\
	1970 & 100  \\
	1971 & 128  \\
	1972 & 160  \\
	1973 & 192  \\		
	\end{tabular} 
	\\ Fonte: dados hipotéticos\\
	Base: R\$200,00
	\end{table}
	
\item Considerando os relativos de base móvel constantes da tabela abaixo, determine:
	\begin{enumerate}
	\item a correspondente série dos relativos de base fixa;
	\item a série dos relativos com base em 1970;
	\item a variação percentual da produção de 1974, em relação às 300 toneladas produzidas em 1972.
	\end{enumerate}
	\begin{table}[!htb]
	\centering
	\caption{Relativos da Produção da Utilidade Y (t)}
	\begin{tabular}{ll}
	Ano & Relativos \\
	\hline 
	1969 & 1,20  \\
	1970 & 1,25  \\
	1971 & 1,20 \\
	1972 & 1,25  \\
	1973 & 1,20  \\		
	1974 & 1,10  \\		
	\end{tabular} 
	\\ Fonte: dados hipotéticos
	\end{table}

\item Considerando os relativos de base móvel constantes da tabela abaixo, determine:
	\begin{enumerate}
	\item a correspondente série dos relativos de base fixa;
	\item a série dos relativos com base em 1971;
	\item sabendo que a produção do ano de 1972 foi de 448 toneladas, a série dos valores originais.
	\end{enumerate}
	\begin{table}[!htb]
	\centering
	\caption{Relativos da Produção da Utilidade Y (t)}
	\begin{tabular}{ll}
	Ano & Relativos \\
	\hline 
	1971 & 1,25 \\
	1972 & 1,12  \\
	1973 & 1,10  \\		
	1974 & 1,50  \\		
	\end{tabular} 
	\\ Fonte: dados hipotéticos
	\end{table}	
	
\item Considerando os relativos de base móvel constantes da tabela abaixo, determine:
	\begin{enumerate}
	\item a correspondente série dos relativos de base fixa;
	\item a série dos relativos com base fixa em 1972;
	\item a variação da tonelagem produzida em 1976 em relação às 330 toneladas produzidas em 1972.
	\end{enumerate}
	\begin{table}[!htb]
	\centering
	\caption{Relativos da Produção da Utilidade Y (t)}
	\begin{tabular}{ll}
	Ano & Relativos \\
	\hline 
	1971 & 1,20 \\
	1972 & 1,25  \\
	1973 & 1,20  \\		
	1974 & 1,50  \\		
	1975 & 1,20  \\	
	1976 & 1,25  \\		
	\end{tabular} 
	\\ Fonte: dados hipotéticos
	\end{table}		

\item Considerando os relativos de base móvel constantes da tabela abaixo, determine:
	\begin{enumerate}
	\item a correspondente série dos relativos de base fixa;
	\item a série dos relativos com base fixa em 1972;
	\item a variação da tonelagem produzida em 1974 em relação às 300 toneladas produzidas em 1972.
	\end{enumerate}
	\begin{table}[!htb]
	\centering
	\caption{Relativos da Produção da Utilidade Y (t)}
	\begin{tabular}{ll}
	Ano & Relativos \\
	\hline 
	1969 & 1,20 \\
	1970 & 1,25 \\
	1971 & 1,20 \\
	1972 & 1,25  \\
	1973 & 1,20  \\		
	1974 & 1,30  \\			
	\end{tabular} 
	\\ Fonte: dados hipotéticos
	\end{table}

\end{enumerate}