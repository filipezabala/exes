\chapter{Regressão e Correlação (tratamento populacional) }

\begin{enumerate}[resume]
%\setcounter{enumi}{1}

\item Considerando os valores de X e Y, referentes a duas variáveis pesquisadas na região A, conforme os dados constantes da seguinte:
	\begin{table}[!htb]
	\centering
	\caption{Determinações das Variáveis Y e X : Região A, 1973}
	\vspace{0.5cm}
	\begin{tabular}{ll}
	Y & X \\
	\hline 
	11 & 10  \\
	14 & 11  \\
	13 & 11  \\
	20 & 15  \\
	15 & 14  \\
	24 & 18  \\
	20 & 20  \\
	27 & 21  \\
	23 & 20  \\
	33 & 30  \\
	\end{tabular}\\
	Fonte: dados hipotéticos\\
	\end{table}\\	
	construa o correspondente diagrama de dispersão e determine:
	\begin{enumerate}
	\item o valor e o significado dos coeficientes de correlação e determinação;
	\item as retas de regressão, procedendo a respectiva representação gráfica;
	\item a estimativa, dada por intervalo, da variável Y para o caso de X = 25;
	\item	a estimativa, dada por intervalo, da variável X para o caso de Y = 30.
	\end{enumerate}


\item Considerando os valores de X e Y, referentes a duas variáveis pesquisadas na região A, conforme os dados constantes da seguinte tabela:
	\begin{table}[!htb]
	\centering
	\caption{Determinações das Variáveis Y e X : Região A, 1977}
	\vspace{0.5cm}
	\begin{tabular}{ll}
	X & Y \\
	\hline 
	18 & 10  \\
	12 & 7  \\
	14 & 8  \\
	10 & 4  \\
	6 & 6  \\
	\end{tabular}\\
	Fonte: dados hipotéticos\\
	\end{table}\\	
	construa o correspondente diagrama de dispersão e determine:
	\begin{enumerate}
	\item o valor e o significado dos coeficientes de correlação e determinação;
	\item as retas de regressão, procedendo a respectiva representação gráfica;
	\item a estimativa, dada por intervalo, da variável Y para o caso de X = 9;
	\item	a estimativa, dada por intervalo, da variável X para o caso de Y = 15;
	\item a variância explicada de Y;	
	\item a variância explicada de X.
	\end{enumerate}

\item Considerando os valores de X e Y, referentes a duas variáveis pesquisadas na região B, conforme os dados constantes da seguinte tabela:
	\begin{table}[!htb]
	\centering
	\caption{Determinações das Variáveis Y e X : Região B, 1975}
	\vspace{0.5cm}
	\begin{tabular}{ll}
	X & Y \\
	\hline 
	32 & 16  \\
	48 & 24  \\
	56 & 28  \\
	64 & 40  \\
	80 & 32  \\
	\end{tabular}\\
	Fonte: dados hipotéticos\\
	\end{table}\\	
	construa o correspondente diagrama de dispersão e determine:
	\begin{enumerate}
	\item o valor e o significado dos coeficientes de correlação e determinação;
	\item as retas de regressão, procedendo a respectiva representação gráfica;
	\item a estimativa, dada por intervalo, da variável Y para o caso de X = 38;
	\item	a estimativa, dada por intervalo, da variável X para o caso de Y = 70;
	\item a variância explicada de Y;	
	\item a variância explicada de X.
	\end{enumerate}
	
\item Considerando os valores de X e Y, referentes a duas variáveis pesquisadas na região K, conforme os dados constantes da seguinte tabela:
	\begin{table}[!htb]
	\centering
	\caption{Determinações das Variáveis Y e X : Região K, 1975}
	\vspace{0.5cm}
	\begin{tabular}{ll}
	X & Y \\
	\hline 
	12 & 18  \\
	8 & 12 \\
	14 & 14  \\
	16 & 10  \\
	20 & 6  \\
	\end{tabular}\\
	Fonte: dados hipotéticos\\
	\end{table}\\	
	construa o correspondente diagrama de dispersão e determine:
	\begin{enumerate}
	\item o valor e o significado dos coeficientes de correlação e determinação;
	\item as retas de regressão, procedendo a respectiva representação gráfica;
	\item a estimativa, dada por intervalo, da variável Y para o caso de X = 16;
	\item	a estimativa, dada por intervalo, da variável X para o caso de Y = 18;
	\item a variância explicada de Y.
	\end{enumerate}
	
\item Considerando Y a tonelagem produzida de certo cereal e X a área plantada em hectares e sabendo que 
$$Yc_{i} = 20 + 1,28X_{i}$$
$$Xc_{i} = 15 + 0,72Y_{i}$$
$$\sigma_{r/Y} = 1,96$$
$$\sigma_{r/X} = 4,41$$
\\ determine
	\begin{enumerate}
	\item o coeficiente de correlação linear;
	\item a estimativa, dada por intervalo, da tonelagem produzida para uma área plantada de 20 hectares;
	\item a estimativa, dada por intervalo, da área cultivada para uma produção de 25 toneladas;
	\item a variância total de Y;
	\item a variância explicada de X.
	\end{enumerate}
	
\item Sabendo que $64\%$ das variações de uma certa variável Y, que possue $\mu_{Y} = 225$ e $\sigma_{T/X} = 5$ são diretamente explicadas pelas correspondentes variações de uma outra variável X, que possue $\mu_{X} = 100$ e $\sigma_{T/X} = 2 $ determine:
	\begin{enumerate}
	\item o coeficiente de correlação linear entre elas;
	\item a estimativa, dada por intervalo, para Y na suposição de X = 40.
	\end{enumerate}

\item Sabendo que $81\%$  das variações de uma certa variável Y, que possue $\mu_{Y}= 38$ e $\sigma_{T/X} = 5$, são diretamente explicadas pelas correspondentes variações de uma outra variável X, que possue $\mu_{X} = 60$ e $\sigma_{T/X} = 10 $ determine:
	\begin{enumerate}
	\item o coeficiente de correlação linear entre elas;
	\item a estimativa, dada por intervalo, para Y na suposição de X = 24.
	\end{enumerate}

\item Sabendo que no estudo do relacionamento das variáveis Y e X foram determinadas as retas de regressão $Yc_{i} = 10 + 0,50X_{i} $ e $Xc_{i} = 12 + 0,98Y_{i} $ caracterize o grau de associação existente entre elas e estabeleça a proporção das variações de uma que é explicada pelas correspondentes variações da outra.

\item Sabendo que no estudo do relacionamento das variáveis Y e X foram determinadas as retas de regressão $Yc_{i} = 10 + 0,36X_{i} $ e $Xc_{i} = 12 + 0,25Y_{i} $ caracterize o grau de associação existente entre elas e estabeleça a proporção das variações de uma que é explicada pelas correspondentes variações da outra.

\item Sabendo que no estudo do relacionamento das variáveis Y e X foi apurado que a variância total de X é igual a 25 e que sua variância explicada é igual a 16, determine o coeficiente de correlação entre X e Y e estabeleça o valor da variância residual de X.

\item Considerando que o coeficiente de correlação entre as variáveis X e Y foi apurado em 0,8 e sabendo que a variância residual de Y foi calculada em 81, determine a variância explicada de Y. 

\item Sabendo que no estudo do relacionamento das variáveis Y e X foram determinadas as retas de regressão $Yc_{i} = 50 + 1,28X_{i} $ e $Xc_{i} = 40 + 0,5Y_{i} $  e considerando, ainda que $\sigma^{2}_{e/Y} = 16 $, determine: 
	\begin{enumerate}
	\item o correspondente coeficiente de correlação e o de determinação;
	\item a variância total e a variância residual de Y;
	\item o intervalo de previsão para Y na hipótese de X = 25.
	\end{enumerate}
	
\item Sabendo que no estudo do relacionamento das variáveis Y e X ffoi apurado que $\sigma^{2}_{T/Y} = 625 $ e que $\sigma^{2}_{r/Y} = 400 $, determine: 
	\begin{enumerate}
	\item o coeficiente de correlação linear existente entre elas e a variância explicada de Y;
	\item considerando que $\mu_{Y} = 500$, $\mu_{X} = 300$ e $\sigma_{T/Y} = 10$ a reta de regressão de Y sobre X;
	\item a estimativa, dada por intervalo, para o caso de X = 200.
	\end{enumerate}	

\item Sabendo que $64\%$ das variações de uma certa variável Y, cuja reta de regressão de Y sobre X é $Yc_{i} = 80 + 0,8X_{i} $ e que $\sigma^{2}_{e/Y} = 16$, estabeleça o correspondente intervalo de estimação para o caso de X = 10.

\end{enumerate}