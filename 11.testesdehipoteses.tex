\chapter{Testes de Hipóteses}

\begin{enumerate}[resume]
%\setcounter{enumi}{1}

\item Supondo que a variável X esteja normalmente distribuída com média $\mu$ e desvio padrão 10, testar a hipótese da $\mu= 30$ contra a hipótese alternativa da $\mu \neq 30$, fixado o nível de significância de 0,05, mediante uma amostra aleatória de tamanho n = 25 na qual $\sum X_{i} = 650$.

\item Supondo que a variável X esteja normalmente distribuída com média $\mu$ e desvio padrão 40, testar a hipótese da $\mu= 200$ contra a hipótese alternativa da $\mu \neq 200$, fixado o nível de significância de 0,05, mediante uma amostra aleatória de tamanho n = 64 na qual $\sum X_{i} = 13.440$.

\item Pelo sistema de avaliação adotado na disciplina de Estatística em certa universidade, face a experiência de longos anos, pode-se afirmar que o grau médio dos alunos que cursam tal materia é de 70 pontos, com um desvio padrão de 30 pontos. Os alunos de determinada turma, em número de 100, obtiveram um somatório de 748 pontos. Fixado o nível de confiança de 0,95, pode-se acreditar que os estudantes dessa turma sejam diferentes dos demais ?

\item Um construtor supoe que a qualidade dos tijolos que utiliza em suas obras está apresentando variações. Em face de experiência anterior, sabe-se que a resistência média ao rompimento de tais tijolos é de 200 quilos, com um desvio padrão de 20 quilos. Numa amostra de 100 tijolos apurou-se que a resistência ao rompimento alcançou um total de 19.600 quilos. Fixado o nível de significância de 0,05, verifique da qualidade não estar apresentando variações.

\item Um fabricante produz determinado artigo cujo tempo de duração, dado em horas,  $N(80 ; 7,5)$. Com o objetivo de aumentar o tempo médio de duraçao de tal produto foi introduzido um novo esquema de fabricação. Para verificar da durabilidade dos artigos produzidos pelo processo foram testadas 225 unidades, obtendo-se uma duração total de 18.184,5 horas. Admitindo que a variabilidade permaneça a mesma e considerando um nível de significância de 0,05, o fabricante deseja saber se o novo processo produz o artigo desejado.

\item Um fabricante de conservas anuncia que o conteudo líquido de uma lata de seu produto é de 2.000 gramas, com um desvio padrão de 40 gramas. A fiscalização de pesos e medidas investigou uma amostra aleatória de 64 latas, verificando que $\sum X_{i}  = 127.360$. Fixado o nível de significância de 0,05, deverá o fabricante ser multado por não efetuar a venda do produto conforme anuncia?

\item Os registros de uma Faculdade mostram que a nota média em determinada disciplina é de 65 pontos, com um desvio padrão de 16 pontos. No último semestre, tendo sido adotado um novo processo de ensino, constatou-se que, em uma amostra de 64 alunos, a nota média foi de 69 pontos. Verificar, ao nível de significância de 0,05, se o novo processo de ensino é de melhor qualidade do que o anterior.

\item Numa amostra de sessenta e quatro lâmpadas produzidas pela empresa ABC, verificou-se que o seu tempo médio de duração foi calculado em 2.485 horas e a sua variância absoluta foi apurada em 6.300. Fixado o nível de significância de 0,05, testar a hipótese da vida média das referidas lâmpadas ser de 2.500 horas contra a hipótese alternativa de sua média ser diferente de 2.500 horas.

\item Um fabricante de conservas anuncia que o conteúdo líquido de uma lata de seu produto é de 1.000 gramas. A fiscalização de pesos e medidas investigou uma amostra aleatória de 100 latas, apurando sua média em 995 gramas e sua variância absoluta em 1584. Fixado o nível de significância de 0,05, deverá o fabricante ser multado por não efetuar a venda conforme anuncia?

\item A Industrial ABC S/A fabricante de um certo equipamento elétrico afirma que a substituição de determinado componente importado pelo similar nacional não diminuiu a durabilidade de seu produto que antes era anunciada como sendo, em média, de 100 horas. Para julgar da aceitabilidade daquela afirmativa, um grande comprador da referida fábrica testou uma amostra de 64 unidades, verificando que $\sum X_{i} = 6.208$ e $\sum X_{i} ^{2} = 618.304$. Fixado o nível de significância de 0,05, estabeleça a conclusão a que chegou o comprador.

\item Certo fabricante afirmava que o tempo médio de duração de seu produto era de 500 horas. Face a utilização de matéria-prima de qualidade inferior, por limitações de mercado, o fabricante supõe que a duração média de seu produto tenha sido alterada. Sabendo que para testar sua opinião, fixado o nível de significância de 0,05, o fabricante investigou uma amostra de 64 unidades, na qual apurou que $\bar{X}= 491$ e $s^{2} = 1.575$, aponte a conclusão obtida pelo referido produtor.

\item Uma organização universitária presume que a aplicação de um determinado método de ensino tenha diminuido de forma expressiva o nível de aproveitamento por parte dos alunos que antes, num sistema de notas de zero a dez, era avaliado como sendo,
em média, equivalente a 5,8. Para julgar a aceitabilidade de tal suposição foi efetuada uma amostra cujas informações permitiram a construção da seguinte\\
	\begin{table}[!htb]
	\centering
	\caption{Distribuição de notas dos alunos integrantes da amostra}
	\begin{tabular}{ll}
	Notas & $f_{i}$ \\
	\hline 
	0 $\vdash$ 2 & 22  \\
	2 $\vdash$ 4 & 30 \\
	4 $\vdash$ 6 & 60  \\
	6 $\vdash$ 8 & 52 \\
	8 $\vdash$ 10 & 36  \\
	\hline 
	$\sum$ & 200
	\end{tabular}
	\end{table}\\
Fixado o nível de significância de 0,05 aponte a conclusão obtida pela referida unidade de ensino. Na solução do presente exercício considere os cálculos somente até a segunda decimal, procedendo o devido arredondamento.

\item	Uma organização comercial presume que a aplicação de nova estratégia mercadológica tenha melhorado de forma expressiva o desempenho de seus vendedores que antes, num sistema de avaliação através da atribuição de escores, era avaliado como sendo,em média, equivalente a 4,82. Para julgar a aceitabilidade de tal suposição foi efetuada uma amostra cujas informações permitiram a construção da seguinte\\
	\begin{table}[!htb]
	\centering
	\caption{Distribuição dos escores integrantes da amostra}
	\begin{tabular}{ll}
	Escores & $f_{i}$ \\
	\hline 
	0 $\vdash$ 2 & 28  \\
	2 $\vdash$ 4 & 43 \\
	4 $\vdash$ 6 & 49  \\
	6 $\vdash$ 8 & 41 \\
	8 $\vdash$ 10 & 39  \\
	\hline 
	$\sum$ & 200
	\end{tabular}
	\end{table}\\
Fixado o nível de significância de 0,05 aponte a conclusão obtida pela referida organização. Na solução do presente exercício considere os cálculos somente até a segunda decimal, procedendo o devido arredondamento.

\item Certo fabricante de parafusos anuncia que $90\%$ de seu produto não apresenta qualquer tipo de defeito. Em uma amostra aleatória de 100 parafusos, apurou-se que 86 não apresentavam defeito. Fixado o nível de significância de 0,05, testar a hipótese da proporção de parafusos perfeitos ser de $90\%$ contra a hipótese alternativa de ser diferente de $90\%$.

\item Admitidas as condições mencionadas no exercício anterior, verifique a hipótese da proporção de parafusos que não apresentam qualquer tipo de defeito ser inferior a $90\%$. 

\item Uma organização universitaria julga que $50\%$ dos estudantes matriculados, em seus diversos cursos, trabalham oito horas por dia. Investigada uma amostra de 100 estudantes, verificou-se que 60 deles trabalhavam oito horas por dia. Fixado o nível de significância de 0,05, deve ser mantida a suposição inicial ? 

\item Admitidas as condições mencionadas no exercício anterior, verifique a hipótese da proporção de estudantes que trabalham oito horas por dia ser superior a $50\%$.

\item Certa organização médica afirma que uma nova vacina e de qualidade superior a até então existente, que é $80\%$ eficaz para curar certa enfermidade num determinado prazo. Examinada uma amostra de 100 pessoas que sofriam da referida doença, constatou-se que no prazo especificado, 86 ficaram curadas. Fixado o nível de confiança de 0,05, verifique da aceitabilidade da afirmativa daquela organização.

\end{enumerate}