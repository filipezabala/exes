\chapter{Aproximação da Normal à Binomial}

\begin{enumerate}[resume]
%\setcounter{enumi}{1}

\item Sabendo que determinado processo industrial produz, em média, $4\%$ de unidades defeituosas e considerando uma amostra ocasional de 150 unidades, determine a probabilidade do número de unidades defeituosas:
	\begin{enumerate}
	\item ser igual a oito;
	\item ser maior do que oito;
	\item	 ser menor do que oito;
	\item ser igual a cinco;
	\item estar contido no intervalo $[5 ; 8]$.
	\end{enumerate}

\item Sabendo que numa prova constituída de 100 questões de escolha simples, cada questão apresenta quatro respostas possiveis, das quais apenas uma é correta, determine a probabilidade de se acertar, ao acaso:
	\begin{enumerate}
	\item vinte questões;
	\item mais de vinte questões;
	\item	 menos de vinte questões;
	\item trinta questões;
	\item de vinte a trinta questões.
	\end{enumerate}

\item Sabendo que $45\%$ das bombas produzidas por uma fábrica de material bélico são classificadas como sendo de porte médio e considerando uma amostra ocasional de 300 bombas, determine a probabilidade de nela se encontrar:
	\begin{enumerate}
	\item menos do que 125 bombas de porte médio;
	\item mais do que 150 bombas de porte médio;
	\item	 150 bombas de porte médio;
	\item 125 bombas de porte médio;
	\item de 125 a 150 bombas de porte médio.
	\end{enumerate}
\end{enumerate}