\chapter{Probabilidade}

\begin{enumerate}[resume]
%\setcounter{enumi}{1}

\item Descreva o espaço amostral associado aos seguintes experimentos aleatórios:
	\begin{enumerate}
	\item jogue uma moeda e observe a face voltada para cima;
	\item jogue um dado e observe o numero mostrado na face voltada para cima;
	\item jogue uma moeda duas vezes e observe a sequência obtida de caras e coroas;
	\item jogue uma moeda duas vezes e observe o número de caras obtido;
	\item jogue uma moeda tres vezes e observe a sequência obtida de caras e coroas;
	\item jogue uma moeda tres vezes e observe o número de caras obtido;
	\item conte o numero de peças fabricadas em certo processo industrial ate que dez peças perfeitas sejam produzidas;
	\item uma caixa contem vinte unidades de certo artigo das quais quatro são defeituosas. Observe o numero de peças extraídas de tal caixa até que se obtenha todas as peças defeituosas;
	\item O tempo de duração, dado em horas, de uma lâmpada conectada a uma fonte elétrica.
	\end{enumerate}
	
\item Supondo o experimento aleatório consistente no duplo lançamento de uma moeda honesta e considerando X como sendo a variável "número de vezes que ocorre a face cara", calcule a probabilidade de:
	\begin{enumerate}
	\item P(X = 0);
	\item P(X = 1);
	\item P(X = 2);
	\end{enumerate}
	
\item Construa um espaço amostral correspondente ao lançamento simultâneo de dois dados honestos e calcule a probabilidade de se obter:
	\begin{enumerate}
	\item uma soma de pontos igual a dez;
	\item um par de valores iguais;
	\item uma soma de pontos maior do que dez;
	\item um par de valores com a primeira componente menor ou igual a três ou com a segunda componente menor ou igual a dois;
	\item uma soma de pontos igual a sete ou dez;
	\item um par de valores com a primeira componente menor ou igual a três e com a segunda componente menor ou igual a dois.
	\end{enumerate}		

\item Uma urna contém seis fichas azuis, quatro fichas brancas e cinco fichas cinzas; 
	\begin{enumerate}
	\item Supondo a extração de uma ficha, calcule a probabilidade de que saía uma ficha:
		\begin{enumerate}
		\item azul;
		\item branca;
		\item cinza;
		\item azul ou cinza;
		\item não azul.
		\end{enumerate}
	\item Supondo a extração sucessiva de três fichas, calcule a probabilidade de que saia:
		\begin{enumerate}
		\item adotado o esquema com reposição, as duas primeiras azuis e a terceira branca;
		\item adotado o esquema sem reposição, as duas primeiras azuis e a terceira branca;
		\item adotado o esquema sem reposição, as três azuis;
		\item adotado o esquema sem reposição, a primeira e a terceira brancas e a segunda azul;
		\item adotado o esquema sem reposição, a primeira azul, a segunda branca e a terceira cinza.
		\end{enumerate}
	\end{enumerate}

\item Supondo o lançamento de um dado correto, calcule a probabilidade de se obter:
	\begin{enumerate}
	\item um ponto múltiplo de dois ou de três;
	\item um ponto múltiplo de dois e de três.
	\end{enumerate}
	
\item Considerando um baralho completo, determine a pro
babilidade de se obter:
	\begin{enumerate}
	\item supondo a extração de uma carta, ou rei ou
espadas;
	\item supondo a extração de duas cartas, adotado o esquema com reposição, uma carta vermelha e uma negra;
	\item supondo a extração de duas cartas, adotado o esquema com reposição, uma carta de espadas e uma dama;
	\item supondo a extração de duas cartas, adotado o esquema sem reposição, duas cartas vermelhas.
	\end{enumerate}	

\item A probabilidade de que certa porta esteja chaveada é igual a 0,8. A chave correspondente a tal porta está em um chaveiro que contém cinco chaves. Se uma pessoa seleciona uma das chaves ao acaso, determine a probabilidade de que a porta seja aberta na primeira tentativa.

\item Sabendo que a probabilidade de que um aluno do sexo feminino obtenha aprovação em um teste de Estatística é de 4/5 e que a de um aluno do sexo masculino é de 2/5, calcule a probabilidade de que:
	\begin{enumerate}
	\item somente o aluno do sexo feminino seja aprovado;
	\item somente o aluno do sexo masculino seja aprovado;
	\item ao menos um dos alunos seja aprovado;
	\item nenhum aluno seja aprovado.
	\end{enumerate}
	
\item A probabilidade de um aluno resolver certo problema é de 1/5 e a de outro aluno é de 5/6. Sabendo que os alunos tentam solucionar o problema independentemente, determine a probabilidade do problema ser resolvido:
	\begin{enumerate}
	\item somente pelo primeiro aluno;
	\item somente pelo segundo aluno;
	\item por ambos;
	\item por nenhum;
	\item por ao menos um dos alunos.
	\end{enumerate}
	
\item A probabilidade de um homem estar vivo daqui a 25 anos é de 3/5 e a de sua mulher é de 5/6. Determine a probabilidade de:
	\begin{enumerate}
	\item ambos estarem vivos;
	\item somente o homem estar vivo;
	\item somente a mulher estar viva;
	\item ao menos um estar vivo;
	\item ao menos um estar morto.	
	\end{enumerate}

\item A peça "A" de um automóvel é produzida por certa fábrica com 80\% de probabilidade de ser perfeita. A peça "B" que na montagem do veículo deve ser ajustada à peça "A", é produzida por outra fábrica com 90\% de probabilidade de ser perfeita. Sabendo que o encaixe de tais peças só é aceitável quando ambas são perfeitas e selecionando-se, ao acaso, uma unidade de cada peça, a probabilidade de se obter um encaixe:
	\begin{enumerate}
	\item aceitável;
	\item inaceitável;
	\item inaceitável por ser somente a peça "A" defeituosa;
	\item inaceitável por ser somente a peça "B" defeituosa;
	\item inaceitável por serem as duas peças defeituosas.
	\end{enumerate}

\item Uma empresa apresenta a probabilidade de erro de data em seus registros igual a 0,04 e de erro por inversão de valores igual a 0,05. Sabendo que o auditor irá apontar a ocorrência de qualquer tipo de incorreção e supondo o exame de 500 documentos, determine o número de documentos que se espera:
	\begin{enumerate}
	\item sejam apontados;
	\item sejam apontados devido somente a erro por inversão de valores;
	\item sejam apontados devido somente a erro de data.
	\end{enumerate}
	
\item Certa máquina apresenta a probabilidade de produzir parafusos com defeito de fenda igual a 0,1 e de parafusos tortos igual a 0,05. Sabendo que o controle de qualidade considera o parafuso inaceitável quando se constata qualquer dos defeitos e supondo o exame de um lote de 5.000 parafusos, determine o número de parafusos que se espera sejam considerados:
	\begin{enumerate}
	\item inaceitáveis;
	\item inaceitáveis devido somente a defeito de fenda;
	\item inaceitáveis por serem somente tortos.
	\end{enumerate}
	
\end{enumerate}  