\chapter{Regressão e Correlação (tratamento amostral) }

\begin{enumerate}[resume]
%\setcounter{enumi}{1}

\item Considerando os valores de X e Y, referentes a duas variáveis amostradas na região A, conforme os dados constantes da seguinte:
	\begin{table}[!htb]
	\centering
	\caption{Determinações das Variáveis Y e X : Região A, 1978}
	\vspace{0.5cm}
	\begin{tabular}{ll}
	X & Y \\
	\hline 
	4 & 16  \\
	6 & 24  \\
	8 & 28 \\
	10 & 32  \\
	12 & 40  \\
	\end{tabular}\\
	Fonte: dados hipotéticos\\
	\end{table}\\	
	construa o correspondente diagrama de dispersão e determine:
	\begin{enumerate}
	\item a estimativa da reta de regressão de Y sobre X;
	\item afixado um nível de significância de 0,05, o teste sobre a aceitabilidade do parâmetro angular como sendo um valor diferente de zero;
	\item o coeficiente de determinação e correlação, interpretando os resultados obtidos;
	\item	a estimativa para Y na hipótese de X = 7;
	\item	a estimativa para Y na hipótese de X = 11.
	\end{enumerate}

\item Considerando os valores de X e Y, referentes a duas variáveis amostradas na região B, conforme os dados constantes da seguinte:
	\begin{table}[!htb]
	\centering
	\caption{Determinações das Variáveis Y e X : Região B, 1978}
	\vspace{0.5cm}
	\begin{tabular}{ll}
	X & Y \\
	\hline 
	5 & 25  \\
	10 & 35  \\
	15 & 65 \\
	20 & 85  \\
	25 & 90  \\
	\end{tabular}\\
	Fonte: dados hipotéticos\\
	\end{table}\\	
	construa o correspondente diagrama de dispersão e determine:
	\begin{enumerate}
	\item a estimativa da reta de regressão de Y sobre X;
	\item afixado um nível de significância de 0,05, o teste sobre a aceitabilidade do parâmetro angular como sendo um valor diferente de zero;
	\item o coeficiente de determinação e correlação, interpretando os resultados obtidos;
	\item	a estimativa para Y na hipótese de X = 12;
	\item	a estimativa para Y na hipótese de X = 22.
	\end{enumerate}

\item Considerando os valores de X e Y, referentes a duas variáveis amostradas na região K, conforme os dados constantes da seguinte:
	\begin{table}[!htb]
	\centering
	\caption{Determinações das Variáveis Y e X : Região K, 1978}
	\vspace{0.5cm}
	\begin{tabular}{ll}
	X & Y \\
	\hline 
	2 & 24  \\
	6 & 21  \\
	10 & 14 \\
	14 & 11  \\
	18 & 5  \\
	\end{tabular}\\
	Fonte: dados hipotéticos\\
	\end{table}\\	
	construa o correspondente diagrama de dispersão e determine:
	\begin{enumerate}
	\item a estimativa da reta de regressão de Y sobre X;
	\item fixado um nível de significância de 0,05, o teste sobre a aceitabilidade do parâmetro angular como sendo um valor diferente de zero;
	\item o coeficiente de determinação e correlação, interpretando os resultados obtidos;
	\item	a estimativa para Y na hipótese de X = 8;
	\item	a estimativa para Y na hipótese de X = 16.
	\end{enumerate}
	
\item Considerando os valores de X e Y, referentes a duas variáveis amostradas na região W, conforme os dados constantes da seguinte:
	\begin{table}[!htb]
	\centering
	\caption{Determinações das Variáveis Y e X : Região W, 1978}
	\vspace{0.5cm}
	\begin{tabular}{ll}
	X & Y \\
	\hline 
	25 & 24  \\
	45 & 36  \\
	65 & 42 \\
	85 & 48  \\
	105 & 60  \\
	\end{tabular}\\
	Fonte: dados hipotéticos\\
	\end{table}\\	
	construa o correspondente diagrama de dispersão e determine:
	\begin{enumerate}
	\item a estimativa da reta de regressão de Y sobre X;
	\item afixado um nível de significância de 0,05, o teste sobre a aceitabilidade do parâmetro angular como sendo um valor diferente de zero;
	\item o coeficiente de determinação e correlação, interpretando os resultados obtidos;
	\item	a estimativa para Y na hipótese de X = 30;
	\item	a estimativa para Y na hipótese de X = 90.
	\end{enumerate}
	
\end{enumerate}